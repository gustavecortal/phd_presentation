% --begin pap's style
\documentclass[10pt]{beamer}

%\usetheme[secheader]{Darmstadt}
%\usetheme{Pittsburgh}

\usepackage[english]{babel}


\usepackage[utf8]{inputenc}
\usepackage[final]{pdfpages}
\usepackage[official]{eurosym}
\usepackage{graphicx}
\usepackage{caption}
\usepackage{csquotes}
\usepackage{multirow}
\usepackage{amsmath}
\usepackage{amsfonts} % for \text
\usepackage{hyperref}
\usepackage{comment}
\usepackage{graphicx}
\usepackage{booktabs}
\usepackage{tabularx}
\usepackage[flushleft]{threeparttable}  % for table notes
\newcommand\mscriptsize[1]{\mbox{\scriptsize\ensuremath{#1}}}
\newcommand\mtiny[1]{\mbox{\tiny\ensuremath{#1}}}
\usepackage{xurl}             % allow line breaks in long URLs
\usepackage{ragged2e}         % better ragged-right with hyphenation
\usepackage{xspace}           % for intelligent spacing after commands
\setlength{\emergencystretch}{3em} % gentle last-resort stretch

\usepackage{xpatch}

% Patch 'itemize' to set itemsep
\xpatchcmd{\itemize}
  {\def\makelabel}
  {\setlength{\itemsep}{1em}\def\makelabel} % Change 1em to your size
  {}
  {}

\usepackage{tikz}
\usetikzlibrary{shapes,backgrounds}

\usepackage{subcaption}

\usepackage[T1]{fontenc}
\usepackage[utf8]{inputenc}
%\usepackage{minted}           % core package
\usepackage{xcolor}           % for background color
\definecolor{LightGray}{gray}{0.95}


\definecolor{blueNCS}{rgb}{0.0, 0.53, 0.74}    % ✓ Commas
\definecolor{bluepanam}{rgb}{0.0, 0.189, 0.79} % ✓ Commas
\definecolor{Aplgreen}{rgb}{0.55, 0.71, 0.0}   % ✓ Commas
\definecolor{DarkBlue}{rgb}{0.1,0.1,0.9}

\hypersetup{
    colorlinks=false,
    linkcolor=bluepanam,
    filecolor=bluepanam,      
    urlcolor=bluepanam,
    pdftitle={Overleaf Example},
    pdfpagemode=FullScreen,
    }

\setbeamercolor{palette primary}{bg=white,fg=black}
\setbeamercolor{palette secondary}{bg=white,fg=bluepanam}
\setbeamercolor{palette tertiary}{bg=white,fg=bluepanam}
\setbeamercolor{palette quaternary}{bg=white,fg=bluepanam}
\setbeamercolor{structure}{fg=bluepanam} % itemize, enumerate, etc
\setbeamercolor{section in toc}{fg=bluepanam} % TOC sections


%% \setbeamercolor{section in head/foot}{fg=white, bg=blue}
\setbeamercolor{title}{fg=bluepanam, bg=white}
\setbeamercolor{author}{fg=black, bg=white}
\setbeamercolor{institute}{fg=black, bg=white}
\setbeamercolor{date}{fg=white, bg=white}


\setbeamercolor{section in head/foot}{fg=bluepanam, bg=white}     
     
\setbeamercolor{author in head/foot}{fg=bluepanam, bg=white}
%\setbeamercolor{author in head/foot}{fg=white, bg=white}
\beamertemplatenavigationsymbolsempty
% --end pap's style

%\newcommand{\customcite}[1]{\citeauthor{#1}, \citetitle{#1}, \citeyear{#1}, \citeurl{#1}}
%\newcommand{\customcitenourl}[1]{\citeauthor{#1}, \citetitle{#1}, \citeyear{#1}}
\newcommand{\parensmall}[1]{{\scriptsize #1}}

% Custom fullcite command without URL, DOI, and urldate
\newcommand{\fullcitenourl}[1]{%
  \AtNextCite{%
    \clearfield{url}%
    \clearfield{doi}%
    \clearfield{urldate}%
  }%
  \fullcite{#1}%
}

\newcommand{\smallcitet}[2][]{\text{\scriptsize \citet[#1]{#2}}} % Simplified version
\newcommand{\smallcitep}[2][]{\text{\scriptsize \citep[#1]{#2}}} % Simplified version


\usepackage{xurl}             % allow line breaks in long URLs
\usepackage{ragged2e}         % better ragged-right with hyphenation
\usepackage{xspace}           % for intelligent spacing after commands
\setlength{\emergencystretch}{3em} % gentle last-resort stretch
\usepackage{appendixnumberbeamer}

%\newcommand{\comment}[1]{}

\usepackage[style=authoryear,natbib=true,backend=biber,date=year,urldate=long]{biblatex}
\addbibresource{phd_slides_biblatex.bib}
\addbibresource{full.bib}
\addbibresource{custom.bib}
\addbibresource{introduction.bib}

% Remove "visited on" dates from URLs
\AtEveryBibitem{\clearfield{urldate}}
\hypersetup{
  colorlinks=true,
  linkcolor=bluepanam,
  filecolor=bluepanam,      
  urlcolor=bluepanam,
  citecolor=black,  % Changed to black to match text color
  pdftitle={Overleaf Example},
  pdfpagemode=FullScreen,
}

% Add more spacing between entries
\setlength\bibitemsep{0.5\baselineskip}

\title{Natural language processing for\\ subjectivity analysis in personal narratives}
\author{Gustave Cortal}
\institute{\footnotesize Thesis director: Alain Finkel \\ Co-advisors: Patrick Paroubek and Lina Ye}
\titlegraphic{%
  \begin{minipage}{\linewidth}
    \centering
    \includegraphics[height=1cm]{img/lmf_logo_emnlp.png}%\hfill
    \hspace{0.5cm}
    \includegraphics[height=1cm]{img/logo_ens_saclay.png}%\hfill
    \hspace{0.5cm}
    %\vfill
    \includegraphics[height=1cm]{img/logo_lisn.png}%\hfill
    \hspace{0.5cm}
    \includegraphics[height=1cm]{img/logo_cnrs.png}
  \end{minipage}%
}
\date{\today}

\makeatletter
\defbeamertemplate*{footline}{myminiframes theme}
  {%
    \begin{beamercolorbox}[colsep=1.5pt]{upper separation line foot}
    \end{beamercolorbox}
    \begin{beamercolorbox}[ht=2.5ex,dp=1.125ex,%
      leftskip=.3cm,rightskip=.3cm plus1fil]{author in head/foot}%
      \leavevmode{\usebeamerfont{author in head/foot}}%
      \hfill%
    %\insertframenumber{}\,/\,\inserttotalframenumber%
    \end{beamercolorbox}%
    \begin{beamercolorbox}[ht=2.5ex,dp=2.125ex,leftskip=.3cm,rightskip=.3cm plus1fil]{section in head/foot}%
      %%      {\usebeamerfont{section in head/foot} somthing written here \hfill  \setlength{\fboxrule}{0pt}\setlength{\fboxsep}{0pt}\fcolorbox{blueNCS}{blueNCS!70}{My own image here}}%
      {\usebeamerfont{section in head/foot} \insertsectionhead \hfill  %\setlength{\fboxrule}{0pt}\setlength{\fboxsep}{0pt}\fcolorbox{blueNCS}{blueNCS!100}{foobar etc}
      }%
      \insertframenumber{}\,/\,\inserttotalframenumber%
    \end{beamercolorbox}%
    \begin{beamercolorbox}[colsep=1.5pt]{lower separation line foot}
    \end{beamercolorbox}
  }
  \makeatother


\makeatletter
\setbeamertemplate{headline}{%
    \begin{beamercolorbox}[wd=\paperwidth,ht=1pt,dp=0pt]{progress bar}
        \begin{tikzpicture}
            % Background of the bar (light gray or light blue)
            \fill[bluepanam!20] (0,0) rectangle (\paperwidth, 1pt);
            % The actual progress based on frame numbers
            \fill[bluepanam] (0,0) rectangle ({(\insertframenumber/\inserttotalframenumber)*\paperwidth}, 1pt);
        \end{tikzpicture}
    \end{beamercolorbox}
}
\makeatother

\begin{document}

\setlength{\parskip}{5pt}%
\setlength{\parsep}{0pt}%
\setlength{\itemsep}{0.25cm}%
\setlength{\leftmargini}{0.5cm}

\begin{frame}
  \titlepage
\end{frame}

\section{Introduction}
\begin{frame}{}
\Large
\vfill
\begin{center}
    Introduction
\end{center}
\vfill
\end{frame}

% Make transition between parts
% 


\begin{frame}{Context}

      \centering
    % Define the size of the diagram
    \begin{tikzpicture}[scale=0.54, every node/.style={scale=0.58}]
    
        % Define transparency and blend mode for nicer overlapping colors
        \begin{scope}[blend group=soft light]
            % Circle 1: Top Left - Computer Science
            \fill[blue!40!white] (135:1.5) circle (2.5);
            
            % Circle 2: Top Right - Machine Learning
            \fill[red!40!white] (45:1.5) circle (2.5);
            
            % Circle 3: Bottom Right - Linguistics
            \fill[green!40!white] (315:1.5) circle (2.5);
            
            % Circle 4: Bottom Left - Psychology
            \fill[orange!40!white] (225:1.5) circle (2.5);
        \end{scope}

        % Labels for the fields
        \node at (135:3.2) [align=center, font=\bfseries] {Computer\\Science};
        \node at (45:3.2)  [align=center, font=\bfseries] {Machine\\Learning};
        \node at (315:3.2) [align=center, font=\bfseries] {Linguistics};
        \node at (225:3.2) [align=center, font=\bfseries] {Psychology};

        % Central Label for Thesis
        % We use a white circle behind the text to make it readable against the mixed colors
        \node [circle, fill=white, inner sep=2pt, font=\bfseries\large, align=center] at (0,0) {This\\thesis};

        % Optional: Draw outlines for better definition
        \draw [blue!80!black, thick] (135:1.5) circle (2.5);
        \draw [red!80!black, thick]  (45:1.5) circle (2.5);
        \draw [green!80!black, thick](315:1.5) circle (2.5);
        \draw [orange!80!black, thick](225:1.5) circle (2.5);

    \end{tikzpicture}



  \begin{itemize}[<+->]
    \item Natural language processing for psychology is underexplored%, despite its fundamental importance for understanding human language,
    \item We build on an existing subfield: sentiment and emotion analysis
    \item We study subjectivity (first-person perspective, meaning-making processes, and experiential content)
    \item We focus on personal narratives (emotional narratives, dream reports)
    %\item Computational approaches to analyzing subjective experience have potential for supporting psychological well-being
    \end{itemize}

    %(\textit{e.g.}, characters and emotions)

\end{frame}


\begin{frame}{Introduction}

How to model subjective experience in personal narratives?

\vspace{0.25cm}
\pause

We first address the \textit{content} by classifying elements of personal narratives, then the \textit{form} through the concept of style

\vspace{0.25cm}
\pause

\begin{itemize}[<+->]
    \item Cognitive science perspective on emotion analysis
    \item French corpus based on emotion components
    \item Emotion analysis in emotional and dream narratives 
    \item Formalization of style in personal narratives
    %\item Automatic thematic analysis in mental health narratives %using language models
\end{itemize}


%\vspace{0.5cm}
%\pause

%\small

%\onslide<7->{
%\textit{2 international conferences (EMNLP, LREC-COLING); 3 international workshops; 2 national conferences and journals (TALN)}}
    
\end{frame}

\section{Contributions}
\begin{frame}{}
\Large
\vfill
\begin{center}
    Contributions
\end{center}
\vfill
\end{frame}

\begin{frame}{Papers}

\textbf{International conferences (2):}
\begin{itemize}[<+->]
  \setlength{\itemsep}{0.5em}
  \item Language models for character and emotion detection in dream narratives \parensmall{(first author, oral, LREC-COLING)}
  \item Formal definition of style in personal narratives \parensmall{(first author, oral, EMNLP)}
\end{itemize}

\vspace{0.25cm}
\pause

\textbf{International workshops (3):}
\begin{itemize}[<+->]
  \setlength{\itemsep}{0.5em}
  \item Data paper on emotional narratives \parensmall{(first author, SIGHUM @ EACL)}
  \item Position paper on emotion analysis \parensmall{(equal contribution, CMCL @ ACL)}
  \item Multimodal emotion analysis competition paper \parensmall{(3rd place, intern supervision, ABAW @ ECCV)}
\end{itemize}

\vspace{0.25cm}
\pause

\textbf{National venues (2, French translations):} 
\begin{itemize}[<+->]
  \setlength{\itemsep}{0.25em}
  \item Position paper on emotion analysis \parensmall{(TALN conference)}
  \item Language models for dream analysis \parensmall{(TAL journal)}
\end{itemize}
  
\end{frame}

\begin{frame}{Open corpus and tools}

\begin{columns}[T,onlytextwidth]
  \begin{column}{0.75\textwidth}
    \textbf{Corpus}:

    \vspace{0.25cm}

    \onslide<1->{\href{https://huggingface.co/datasets/gustavecortal/FrenchEmotionalNarratives}{French narratives based on emotion components}}

    \vspace{0.5cm}
    

    \onslide<2->{\textbf{Tools}:

    \vspace{0.25cm}

    \href{https://huggingface.co/gustavecortal/dream-t5}{Language model for emotion and character prediction in dream narratives} \parensmall{+400 downloads}

    \vspace{0.5cm}
    

    French language models for \href{https://huggingface.co/gustavecortal/distilcamembert-cae-component}{emotion component prediction} and \href{https://huggingface.co/gustavecortal/distilcamembert-cae-all}{discrete emotion prediction} \parensmall{+1200 downloads}}

    \vspace{0.25cm}

    \footnotesize
    Models were trained using the Jean Zay supercomputer
  \end{column}
  \begin{column}{0.20\textwidth}
    \centering
    \includegraphics[width=\linewidth]{img/logo_huggingface.png}
    \textbf{\footnotesize{\href{https://huggingface.co/gustavecortal}{hf.co/gustavecortal}}}
  \end{column}
\end{columns}

\vspace{0.5cm}

\footnotesize
%Models were trained using the Jean Zay supercomputer

Licence: CC-BY-NC 4.0
  
\end{frame}

\section{Cognitive science perspective on emotion analysis}

\begin{frame}{}
    \vfill
    \begin{center}
        \Large Cognitive science perspective on emotion analysis
    \end{center}
    \vfill

    \footnotesize
    \textbf{G. Cortal} and C. Bonard. \href{https://aclanthology.org/2024.cmcl-1.23/}{Improving Language Models for Emotion Analysis: Insights from Cognitive Science}. \textit{CMCL @ ACL 2024}.
\end{frame}

\begin{frame}{Psychology and emotion annotation}

%What are current limitations and interesting research directions?

%\pause
%\vspace{0.25cm}

%We review psychological theories of emotion and emotion annotation schemes in NLP

%\pause
%\vspace{0.25cm}

%What are current limitations? 

%[add refs to each theory and annotation schemes]

%[add refs to klinger, etc.]
{\small
\begin{center}
\begin{tabular}{p{6.5em}p{10.5em}p{13.5em}}
\textbf{Psychological theories} & \textbf{In text, emotion is...} & \textbf{Example} \\
\hline
\addlinespace
Basic emotions theory & a \textit{category} & "I love philosophy." $\rightarrow$ \texttt{joy} \\
\addlinespace
\smallcitet{darwin_expression_1872,tomkins_affect_1962,ekman_basic_1999,plutchik} & & \\
\smallcitet{demszky-etal-2020-goemotions,greschner2025categoricalemotionsappraisals} & & \\ 
\addlinespace
\hline
\addlinespace
\pause
Constructivist theories & a continuous value with an \textit{affective} meaning & "His voice soothes me." $\rightarrow$ \texttt{valence} (4/5), \texttt{arousal} (1/5) \\
\addlinespace
\smallcitet{schachter_cognitive_1962,russell_core_1999} & & \\
\smallcitet{buechel-hahn-2017-emobank} & & \\
\addlinespace
\hline
\addlinespace
\pause
Appraisal theory& a continuous value with a \textit{cognitive} meaning & "I received a surprise gift." $\rightarrow$ \texttt{sudden} (4/5), \texttt{control} (0/5) \\ 
\smallcitet{arnold_emotion_1960,lazarus_progress_1991}  & & \\
\smallcitet{Troiano2023} & & \\
%\addlinespace
%\hline
%\pause
%& composed of \textit{semantic roles} & "Louise (\texttt{experiencer}) was angry (\texttt{cue}) towards Paul (\texttt{target}), because he didn’t inform her (\texttt{cause})." \\
%\smallcitet{lee-etal-2010-text,campagnano-etal-2022-srl4e,klingerWhereAreWe2023} & & \\
\end{tabular}
\end{center}
}

%This section will present the three main emotion theories in psychology to provide a background for connecting emotion analysis in natural language processing with cognitive science.
%\paragraph{Basic emotion theory.} Basic emotion theory is certainly the most influential today. Inspired by Darwin's research on emotions \parencite{darwin_expression_1872}, it postulates a certain number of discrete, basic emotions that are universal and innate among humans due to their evolutionary origins. Emotions are understood as psycho-physiological \enquote{programs} that were naturally selected to help overcome recurrent evolutionary challenges \parencite{cosmides_evolutionary_2000}. A prominent version is that of Paul Ekman \parencite{ekman_basic_1999}, who sought to show, as Darwin envisaged, that some emotions are expressed with the same facial expressions across cultures. Ekman used Darwin's \parencite{darwin_expression_1872} list of six \enquote{core} expressions of emotions: anger, fear, surprise, disgust, happiness, and sadness. He notably conducted studies with individuals having no exposure to Western culture, indicating that they could accurately identify facial expressions for these six emotions \parencite{ekman1971constantsac}. It should be noted that Ekman left it open how many basic emotions there are. Besides the six emotions listed, candidates include amusement, contempt, embarrassment, guilt, pride, and shame \parencite{ekman_basic_1999}. Other versions of basic emotion theory have different lists \parencite{tomkins_affect_1962,izard_basic_1992,panksepp_affective_1998,plutchik}. For a discussion of the evidence supportive of basic emotion theory, notably the potential physiological and neurological signatures of basic emotions, see \citet[129––131]{moors_demystifying_2022}.

%\paragraph{Psychological constructivism.} Psychological constructivism is the most influential alternative to basic emotion theory today. It rejects that there are discrete, basic emotions universally shared by humans and posits instead that emotion kinds such as anger, fear, and joy are constructed through the interplay of biological, psychological, and sociocultural factors. Early proponents include \citet{schachter_cognitive_1962}, but its main representatives are James Russell and Lisa Feldman Barrett \parencite{russell_core_1999}. Psychological constructivists focus on the feeling component of emotions that they interpret as a continuum with no categorical barriers. Feelings are typically represented in a two-dimensional space with a valence axis (pleasant to unpleasant) and an arousal axis (activation to deactivation). The impression that there are discrete emotions is seen as a social construct: different forms of enculturation yield different ways to conceptualize or label our bodily feelings into discrete emotional kinds. For a discussion of the evidence supportive of psychological constructivism, see \citet[261––265]{moors_demystifying_2022}. Some evidence comes from so-called \enquote{arousal misattribution} studies, i.e. cases where subjects misinterpret the source of their arousal and where that seems to influence what emotions they undergo.

%\paragraph{Appraisal theory.} The third major psychological theory of emotion is appraisal theory, whose empirical version was pioneered by Magda Arnold \parencite{arnold_emotion_1960}. It was developed to explain the absence of a bijective, one–to–one correspondence between kinds of emotions and emotional stimuli, \textit{i.e.}, the fact that the same kind of stimuli triggers different emotions and that different kinds of stimuli trigger the same kind of emotion. To explain this fact, appraisals are postulated as mediators between stimuli and emotional reactions. Appraisals are cognitive evaluations (unconscious, fast, and error-prone) of the relevance of stimuli given one's concerns and how one should react. Appraisal theory hypothesizes that, for instance, Sam is fearful of the mouse in the kitchen because he appraises it as an imminent threat to his safety, while Maria, on the other, is angry that there is a mouse in the kitchen because she appraises it as an intruder to be kicked out. Thus, each emotion kind can be analyzed by the associated appraisal. For instance, \citet{lazarus_progress_1991} proposes \textit{imminent danger} for fear, \textit{demeaning offense} for anger, \textit{irrevocable loss} for sadness, and \textit{progress towards a goal} for happiness. 

%\paragraph{Emotion consists of semantic roles.} An emotion cannot be reduced to a category or continuous values with affective or cognitive meaning. To better understand an emotional event, several approaches associate spans of text with semantic roles, such as \textit{cause}, \textit{target}, \textit{experiencer}, and \textit{cue} of the emotion \parencite{lee-etal-2010-text,kim-klinger-2019-analysis,bostan-etal-2020-goodnewseveryone,oberlanderExperiencersStimuliTargets2020,campagnano-etal-2022-srl4e,weggeExperiencerSpecificEmotionAppraisal2023,cortal-2024-sequence}. Thus, instead of considering emotion as caused by an event, semantic role labeling of emotions considers that emotion \textit{is} an event \parencite{klingerWhereAreWe2023} that must be reconstructed by answering the question: \enquote{Who (\textit{experiencer}) feels what (\textit{cue}) towards whom (\textit{target}) and why (\textit{cause})?}. In this example, each text span can be associated with a semantic role: \enquote{Louise (\textit{experiencer}) was angry (\textit{cue}) at Paul (\textit{target}) because he did not warn her (\textit{cause}).}.


%\vspace{0.5cm}

%\scriptsize

%\textbf{G. Cortal} and C. Bonard. \href{https://aclanthology.org/2024.cmcl-1.23/}{Improving Language Models for Emotion Analysis: Insights from Cognitive Science}. \textit{CMCL, ACL 2024}.
    
\end{frame}

\begin{frame}{Psychology and emotion annotation}

{\small
\begin{center}
\begin{tabular}{p{6.5em}p{10.5em}p{13.5em}}
\textbf{Psychological theories} & \textbf{In text, emotion is...} & \textbf{Example} \\
\addlinespace
\hline
& composed of \textit{semantic roles} & "Louise (\texttt{experiencer}) was angry (\texttt{cue}) towards Paul (\texttt{target}), because he didn’t inform her (\texttt{cause})." \\
\smallcitet{campagnano-etal-2022-srl4e,klingerWhereAreWe2023} & & \\
\smallcitet{lee-etal-2010-text,xia-ding-2019-emotion,tammewar-etal-2020-annotation} & & \\
\end{tabular}
\end{center}
}

\vspace{0.5cm}
\pause

%\paragraph{Emotion is a refined feeling.} Sentiment analysis, a fundamental task in natural language processing, is sometimes considered a simplified version of emotion analysis. In its most basic form, sentiment analysis associates textual units with a category indicating a polarity (\textit{positive} or \textit{negative}) \parencite{poriaTipIcebergCurrent2020a}. A finer-grained task identifies aspects of a product or topic and determines the sentiment expressed about each of these aspects \parencite{zhangSurveyAspectBasedSentiment2022}. For example, in the sentence \enquote{The battery life of this phone is amazing, but its camera quality is disappointing.}, the sentiment is \textit{positive} for the aspect \enquote{battery life} and is \textit{negative} for the aspect \enquote{camera quality}.

%Sentiment analysis, a fundamental task in natural language processing, is sometimes considered a simplified version of emotion analysis. In its most basic form, sentiment analysis associates textual units with a category indicating a polarity (\textit{positive} or \textit{negative}) \parencite{poriaTipIcebergCurrent2020a}

Similar to aspect-based sentiment analysis \smallcitep{zhangSurveyAspectBasedSentiment2022}: \enquote{The battery life is \textit{amazing} (\textcolor{Aplgreen}{\large$\boldsymbol{+}$}), but its camera quality is \textit{disappointing} (\textcolor{red!70!black}{\large$\boldsymbol{-}$}).}
%\vspace{0.2cm}

%\centering
%\textcolor{Aplgreen}{\Large$\boldsymbol{+}$}\,\textit{positive}\hspace{1cm}
%\textcolor{red!70!black}{\Large$\boldsymbol{-}$}\,\textit{negative}

    
\end{frame}


\begin{frame}{Limitations in emotion analysis}

\begin{itemize}[<+->]
    %\item Different emotion theories lead to divergences in how to annotate them in the text [add refs]
    \item Though the theories reviewed are usually considered rivals, their integration is possible and desirable \smallcitep{schererTheoryConvergenceEmotion2022}
    \item Emotion verbalization is underexplored \smallcitep{micheli2013,etienne-etal-2022-psycho}
    \item Benchmarks evaluate certain aspects of emotional understanding but do not consider its full complexity \smallcitep{campagnano-etal-2022-srl4e,zhangSentimentAnalysisEra2023,paechEQBenchEmotionalIntelligence2024}
\end{itemize}

%Recent benchmarks evaluate language models on specific aspects of emotional understanding \parencite{wangEmotionalIntelligenceLarge2023,paechEQBenchEmotionalIntelligence2024}, but they don't consider its full richness \parencite{scherer_componential_2007,mayer_human_2008,oconnor_measurement_2019}. For example, \citet{paechEQBenchEmotionalIntelligence2024} assesses emotional understanding by predicting the intensity of multiple emotions in conflict scenes. Some benchmarks evaluate models on related tasks, such as sentiment analysis \parencite{zhangSentimentAnalysisEra2023} and theory of mind \parencite{zhouHowFaRAre2023,maHolisticLandscapeSituated2023,kimFANToMBenchmarkStresstesting2023,gandhiUnderstandingSocialReasoning2023}. However, no benchmark specifically proposes to evaluate the multiple facets of emotions that affective sciences reveal (Section \ref{theorie_emotion}). Therefore, it is difficult to know whether current models are efficient for emotional understanding.

%This limitation is compounded by the fact that it is difficult to clearly determine which properties of emotional understanding are to be evaluated. We believe that evaluating language models should be grounded in research on human emotional communication, especially psycholinguistics. For example, before the age of ten, basic emotions (\textit{e.g.}, joy or sadness) are better remembered than complex emotions (\textit{e.g.}, pride or guilt) \parencite{davidsonChildrenRecallEmotional2001,creissenQuelleRepresentationDifferentes2017a}. From six to ten years old, \textit{labeled} emotions are better understood than \textit{suggested} emotions \parencite{blancComprehensionContesEntre2010,creissenQuelleRepresentationDifferentes2017a}. Another example of relevant studies concerns the difficulty that autistic people have in understanding different types of emotional expressions \parencite{ball_conversational_2024}. These studies suggest that, for humans, different types of emotions and different modes of emotional expression are more or less difficult to interpret. It would be desirable for benchmarks to evaluate language models in ways that reflect the relative difficulty of tasks for humans. Such a project would certainly benefit from research in cognitive pragmatics (Section \ref{modele_detective}), knowing, for example, that people with communication disorders have difficulty understanding conversational implicatures \parencite{ball_conversational_2024}, which indicates that the different sources of evidence distinguished by the detective analysis are associated with different levels of difficulty.

%\cite{ekman_basic_1999}, \cite{russell_circumplex_1980,barrett_how_2017}, \cite{arnold_emotion_1960}

\end{frame}

\section{Linguistic and cognitive science theories}

\begin{frame}{}
\Large
\vfill
\begin{center}
    Linguistic and cognitive science theories
    %\section{Cognitive science perspective on emotion analysis}
\end{center}
\vfill

\end{frame}

\begin{frame}{Which verbal signs are used to infer expressed emotions?}

%Which verbal signs are used to infer expressed emotions?

%\pause
%\vspace{0.5cm}

Raphaël Micheli categorizes a range of linguistic markers into three \textit{emotion expression modes} \smallcitep{micheliEsquisseDuneTypologie2013}. The emotion can be: 

%\cite{Micheli2014}
\pause
\vspace{0.5cm}

\begin{itemize}[<+->]
    \item \textit{labeled} explicitly with an emotional term ("I am \underline{sad}")
    \item \textit{shown} with utterance features such as interjections and punctuations ("\underline{Ah!} That's great\underline{!}")
    \item \textit{suggested} with the description of a situation which generally, in a given sociocultural context, leads to an emotion ("\underline{She gave me a gift}")
\end{itemize}

\pause
\vspace{0.5cm}

$\rightarrow$ Emotion expression modes vary in interpretive difficulty \smallcitep{blancComprehensionContesEntre2010,creissenQuelleRepresentationDifferentes2017a,ball_conversational_2024}

%[add refs psycholinguistic, psychiatry, refs aline etienne]

%For example, before the age of ten, basic emotions (\textit{e.g.}, joy or sadness) are better remembered than complex emotions (\textit{e.g.}, pride or guilt) \parencite{davidsonChildrenRecallEmotional2001,creissenQuelleRepresentationDifferentes2017a}. From six to ten years old, \textit{labeled} emotions are better understood than \textit{suggested} emotions \parencite{blancComprehensionContesEntre2010,creissenQuelleRepresentationDifferentes2017a}

%Another example of relevant studies concerns the difficulty that autistic people have in understanding different types of emotional expressions \parencite{ball_conversational_2024}

\pause
\vspace{0.25cm}

$\rightarrow$ There exist an annotation scheme for emotion expression modes \smallcitep{etienne-etal-2022-psycho,dragos-etal-2022-angry}

\end{frame}

\begin{frame}{How to integrate psychological theories of emotion?}

%How to integrate psychological theories of emotion?

%\pause
%\vspace{0.5cm}

%I use the \textbf{integrated framework for emotion theories} (Scherer, 2022):

\begin{figure}
    \centering
    \includegraphics[width=0.9\linewidth]{img/scherer_integrated_framework.png}
    \caption{\textbf{The integrated framework for emotion theories} \smallcitep{schererTheoryConvergenceEmotion2022a}. Rectangles represents the components constituting an emotional episode, and arrows represent causation.}%Emotional episodes are synchronized changes in four components \smallcitep{schererTheoryConvergenceEmotion2022a}.}
    \label{fig:placeholder}
\end{figure}

%[refs cortal et al.]

%[add Appraisal dimensions]

\pause

$\rightarrow$ We use this framework to construct a corpus based on components
    
\end{frame}

\begin{comment}

  \begin{frame}{Research directions}
  \begin{itemize}
    \item Construct a \textbf{unified annotation scheme} to capture the emotional phenomenon better and benefit from knowledge transfer between tasks
    \item Build \textbf{benchmarks} that evaluate various aspects of the emotional phenomenon [based on human difficulty]% based on emotion expression modes in linguistics and detective analysis in pragmatics
    \item Develop new \textbf{methods} in natural language processing [be more precise]% based on detective analysis to improve performance and explainability of models
\end{itemize}
\end{frame}

\end{comment}

\section{French narratives based on emotion components}

\begin{frame}{}
\Large
\vfill
\begin{center}
    French narratives based on emotion components
    %\section{Construction of an emotion dataset}

    \vfill
\end{center}

\footnotesize{
%Available at \href{https://huggingface.co/datasets/gustavecortal/FrenchEmotionalNarratives}{hf.co/datasets/gustavecortal/FrenchEmotionalNarratives} 

The corpus is available at \href{https://huggingface.co/datasets/gustavecortal/FrenchEmotionalNarratives}{hf.co/datasets/gustavecortal/FrenchEmotionalNarratives}

\vspace{0.5cm}

\textbf{G. Cortal}, A. Finkel, P. Paroubek, L. Ye. \href{https://aclanthology.org/2023.latechclfl-1.8/}{Emotion Recognition based on Psychological Components in Guided Narratives for Emotion Regulation}. \textit{SIGHUM @ EACL 2023}.}% \href{https://underline.io/lecture/71953-emotion-recognition-based-on-psychological-components-in-guided-narratives-for-emotion-regulation}{video}.
\end{frame}

\begin{frame}{Motivation}

\textbf{Limitation}: Existing datasets do not consider all emotion components

\vspace{0.5cm}
\pause

\citet{kim-klinger-2019-analysis} study emotion communication in fan-fiction via sensations, postures, and facial expressions

\vspace{0.5cm}
\pause

\citet{tammewar-etal-2020-annotation} annotate emotion carriers (events, people, objects) in spoken personal narratives in German \smallcitep{rathner18b_interspeech}

\vspace{0.5cm}
\pause

\citet{casel-etal-2021-emotion} associate text spans with Scherer's emotion components in literature and Twitter corpora

\vspace{0.5cm}
\pause

$\rightarrow$ New French corpus of emotional narratives structured by the authors according to their behaviors, thoughts, physical feelings, and reasons
    
\end{frame}

\begin{frame}{Cognitive Analysis of Emotions}


\textbf{Goal}: Create a corpus of narratives structured according to emotion components, following a questionnaire from \textit{Cognitive Analysis of Emotions} \smallcitep{Finkel2022}

\vspace{0.25cm}
\pause

The questionnaire:

\begin{minipage}{.78\linewidth}
\begin{itemize}[<+->]

\item explores emotions with behavioral (\textit{behavior}), physiological (\textit{feeling}), and cognitive (\textit{thinking} and \textit{reason)} components
\item uses emotion components to reorganize the narrative of experienced events
\item helps individuals better regulate their emotions
    
    
\end{itemize}

\end{minipage}
\hfill
\begin{minipage}{.20\linewidth}
\centering
\includegraphics[scale=0.23]{img/manuel.jpg}
\smallcitet{Finkel2022}
\end{minipage}
  

\end{frame}

\begin{frame}{French narratives based on emotion components}

\textbf{Contribution}: +1,000 narratives structured with emotion components by the writers themselves plus discrete emotion labels

%\vspace{0.05cm}
%\pause

\begin{table}
    \centering
    \resizebox{1\textwidth}{!}{
\begin{tabular}{l|p{0.78\textwidth}}
 
                   \textbf{Component} &
                 \textbf{Answer} \\
 
\hline
          \texttt{Behavior} & I'm giving a lecture on a Friday morning at 8:30. A student goes out and comes back a few moments later with a coffee in his hand. \\
\texttt{Feeling} & My heart is beating fast, and I freeze, waiting to know how to act. \\
  \texttt{Thinking} & I think this student is disrupting my class. \\
\texttt{Reason} & The student attacks my ability to be respected in class. \\

\end{tabular}}
%\captionof{table}{Example of an emotional narrative structured according to emotion components. More than 1,000 narratives were collected using emotion regulation questionnaires.}
\label{tab:description_corpus}
\end{table} % open-ended questions, my thesis director and I

\small
\vspace{0.25cm}

Chosen emotion: \texttt{anger} (possible choices: \texttt{anger}, \texttt{fear}, \texttt{joy}, \texttt{sadness})

\vspace{0.25cm}
\pause

$\rightarrow$ A. Finkel has been collecting narratives since 2005 during emotion regulation sessions; I structured them to build a corpus for emotion classification

%\vspace{0.5cm}
%\small
%1000 narratives were collected during emotion regulation sessions
%\vspace{0.5cm}

%\scriptsize

%\textbf{G. Cortal}, A. Finkel, P. Paroubek, L. Ye. \href{https://aclanthology.org/2023.latechclfl-1.8/}{Emotion Recognition based on Psychological Components in Guided Narratives for Emotion Regulation}. \textit{SIGHUM, EACL 2023}.% \href{https://underline.io/lecture/71953-emotion-recognition-based-on-psychological-components-in-guided-narratives-for-emotion-regulation}{video}.
    
\end{frame}

\section{Language models for emotion analysis}

\begin{frame}{}
\Large
\vfill
\begin{center}
    Language models for emotion analysis in emotional and dream narratives
    %\section{Training language models for emotion analysis}
\end{center}

\vfill

\footnotesize

%Generated dream narratives available on \href{https://gustavecortal.com/project/oneirogen}{gustavecortal.com}.

%\href{https://huggingface.co/gustavecortal/oneirogen-7B}{Oneirogen}, a language model for dream generation along with \href{https://huggingface.co/datasets/gustavecortal/DreamBank-annotated}{27,000 annotated dream narratives}

%\href{https://huggingface.co/gustavecortal/dream-t5}{Dream-T5}, a language model for emotion and character prediction in dream narratives

Language models are available on \href{https://huggingface.co/gustavecortal}{hf.co/gustavecortal}

\vspace{0.5cm}

\textbf{G. Cortal}, A. Finkel, P. Paroubek, L. Ye. \href{https://aclanthology.org/2023.latechclfl-1.8/}{Emotion Recognition based on Psychological Components in Guided Narratives for Emotion Regulation}. \textit{SIGHUM @ EACL 2023}

\vspace{0.5cm}

\textbf{G. Cortal}. \href{https://aclanthology.org/2024.lrec-main.1282/}{Sequence-to-Sequence Language Models for Character and Emotion Detection in Dream Narratives}. \textit{LREC-COLING 2024}

\end{frame}

\begin{frame}{Discrete emotion detection based on components}

%\textbf{Goal}: Discrete emotion detection based on components
%\vspace{0.5cm}

%"Le choix a été fait de comparer l'approche traditionnelle basée sur le lexique (TF-IDF) avec l'état de l'art actuel (modèles contextuels).

%une étape intermédiaire avec Word2Vec aurait permis de décorréler précisément ce qui relève de la sémantique statique de ce qui relève de la sémantique contextuelle.

\begin{table}
  \centering
  \resizebox{1\textwidth}{!}{
\begin{tabular}{l|ccccccc}

&\multicolumn{3}{c}{\textbf{Logistic Regression} (tf-idf)}&\multicolumn{3}{c}{\textbf{CamemBERT}} \\

           \textbf{Component} &  Precision &     Recall &   $F_1$ &  Precision &     Recall &   $F_1$ \\
\hline
        All  & 71.2\,\mscriptsize{(2.6)} & 69.1\,\mscriptsize{(2.2)} & 67.8\,\mscriptsize{(2.3)} & \textbf{85.1} & \textbf{84.8} & \textbf{84.7} \\
        \pause
        Without \texttt{behavior}   & 77.4\,\mscriptsize{(2.3)} & 75.8\,\mscriptsize{(2.4)} & 74.5\,\mscriptsize{(2.6)} & 80.3 & 79.8 & 79.7 \\
        Without \texttt{feeling}  & 64.3\,\mscriptsize{(1.9)} & 61.5\,\mscriptsize{(1.2)} & 61.3\,\mscriptsize{(2.2)} & 81.6 & 79.8 & 79.9  \\
        Without \texttt{thinking}  & 70.9\,\mscriptsize{(1.8)} & 69.1\,\mscriptsize{(2.0)} & 68.3\,\mscriptsize{(2.2)} & 79.6 & 78.5 & 78.7 \\
        Without \texttt{reason}  & 64.3\,\mscriptsize{(4.1)} & 64.5\,\mscriptsize{(2.4)} & 62.3\,\mscriptsize{(2.8)} & 78.7 & 78.5 & 78.6 \\
        \pause
      Only \texttt{behavior}  & 52.1\,\mscriptsize{(3.5)} & 54.6\,\mscriptsize{(2.9)} & 51.7\,\mscriptsize{(2.9)} & 68.4  & 67.1 & 66.6 \\
Only \texttt{feeling}  & 69.6\,\mscriptsize{(1.5)} & 68.9\,\mscriptsize{(2.1)} & 68.4\,\mscriptsize{(2.0)} & 67.8 & 68.4 & 67.7 \\
  Only \texttt{thinking}  & 50.1\,\mscriptsize{(3.4)} & 53.8\,\mscriptsize{(2.3)} & 50.6\,\mscriptsize{(2.7)} & 70.5 & 70.1 & 70.1 \\
        Only \texttt{reason}  & 68.2\,\mscriptsize{(1.8)} & 66.8\,\mscriptsize{(2.2)} & 66.6\,\mscriptsize{(2.3)} & 71.4 & 68.4 & 68.9 \\
\end{tabular}}
  %\caption{Scores (± std) for discrete emotion classification based on components.}
  \label{tab:pred_emotion}
\end{table}

\vspace{0.25cm}

% We use \onslide<4-> to force this text to wait until the table is finished (slides 1-3)
\onslide<4->{
$\rightarrow$ All components help; best results come from using all, \textit{supporting Scherer’s hypothesis}
}

\vspace{0.25cm}

% This appears on the next click (slide 5)
\onslide<5->{
$\rightarrow$ Some components benefit from contextual understanding and world knowledge; need to study static \textit{vs.} contextual semantics
}

\end{frame}

\begin{frame}{Motivation for dream analysis}

  We performed emotion analysis on concrete, real life situations

  \vspace{0.5cm}
  \pause

  We now turn to oniric, fictional situations: dream narratives

  \vspace{0.5cm}
  \pause

  According to the \textit{continuity hypothesis}, dreams reflect waking-life concerns, emotions, and social contexts \smallcitep{continuity}

  \vspace{0.5cm}
  \pause

  $\rightarrow$ Dream narratives possess a narrative structure and represent attempts to communicate subjective experience

\end{frame}

\begin{frame}{Quantitative analysis of dream narratives}

  %We apply our framework to dream narratives as they possess a narrative structure and represent attempts to communicate subjective experience


%Need other datasets with narrative structure, emotional content, and available for research

%\vspace{0.5cm}
%\pause

\textit{Quantitative dream analysis} studies the continuity hypothesis, and relies on dream databases and annotation schemes \smallcitep{dreamscale,dreambank}

\vspace{0.5cm}
\pause

DreamBank contains 27,000 narratives, only 1823 annotated with the Hall and Van de Castle (HVdC) scheme \smallcitep{hvdc,dreambank}

\vspace{0.5cm}
\pause

The annotation process is complex and costly

\vspace{0.5cm}
\pause

$\rightarrow$ How to automate the annotation process using language models?

%side project in my PhD
    
\end{frame}

\begin{frame}{Example of an annotated dream with HVdC}

  %

  %\vspace{0.25cm}

\begin{figure}
    \centering
    \includegraphics[width=1.\linewidth]{img/hdvc_girl.png}
    %caption{\textbf{Example of an annotated dream narrative with HVdC.}}
    \label{fig:placeholder}
\end{figure}

\footnotesize{

\vspace{-0.5cm}
\textbf{Character}:

\begin{itemize}\setlength{\itemsep}{0.2em}
  \item \textbf{Status}: individual alive (\textbf{1}), group alive (\textbf{2}), dead individual (\textbf{3}), dead group (\textbf{4}), imaginary individual (\textbf{5}), imaginary group (\textbf{6}), original form (\textbf{7}), changed form (\textbf{8})
  \item \textbf{Gender}: male (\textbf{M}), female (\textbf{F}), joint (\textbf{J}), indefinite (\textbf{I})
  \item \textbf{Identity}: known (\textbf{K}), prominent (\textbf{P}), occupational (\textbf{O}), ethnic (\textbf{E}), unknown (\textbf{U})
  \item \textbf{Age}: adult (\textbf{A}), teen (\textbf{T}), child (\textbf{C})
\end{itemize}

\textbf{Emotion}: anger (\textbf{AN}), apprehension (\textbf{AP}), sadness (\textbf{SD}), confusion (\textbf{CO}), and happiness (\textbf{HA})}
  
\end{frame}

\begin{frame}{Existing research on computational dream analysis}

  \textit{Lexical-based approaches} associate text spans with specific categories (\textit{e.g.}, type of interactions) \smallcitep{miller-1994-wordnet,ourdreams}

  \vspace{0.5cm}
  \pause

  \textit{Distributional semantic-based approaches} represent text spans in a vector space  to identify prototypical situations \smallcitep{GUTMANMUSIC2022103428}

  \vspace{0.5cm}
  \pause

  \citet{McNamara_Duffy-Deno_Marsh_Marsh_2019,Yu2022} combine the lexical-based and distributional semantic-based approaches with machine learning% to compare predictions with established coding systems

  %Overall, prior work focuses on emotion presence. 




  %\citet{ourdreams} use the lexical database WordNet \citep{miller-1994-wordnet} to identify some aspects of the Hall and Van de Castle (HVdC) scheme, such as friendly and aggressive interactions.

  %The distributional semantic-based approach uses models for the vector representation of words or phrases. \citet{GUTMANMUSIC2022103428} identify prototypical situations of flight or attack through embedding and clustering narratives in a vector space with a pre-trained Sentence-BERT model 

  %Previous studies present two major limitations: a partial consideration of the context of the narratives and the absence of comparison with established coding systems like the HVdC scheme. To remedy these limitations, some studies combine the dictionary-based or distributional semantic-based approaches with machine learning \citep{McNamara_Duffy-Deno_Marsh_Marsh_2019,Yu2022}. For example, \citet{Yu2022} combines a sentiment dictionary with a support vector machine \citep{cortes1995support} to predict the overall sentiment of a dream narrative. \citet{bertolini2023automatic} are the first to fine-tune a pre-trained language model for detecting the absence or presence of emotions in narratives. This approach makes it possible to consider the whole context of the narrative and evaluate performance by comparing the model's predictions with HVdC gold-standard annotations. However, this study does not account for the frequency with which an emotion appears in the narrative. Moreover, emotions are not identified with the characters experiencing them. 

  %Generally, the studies we have cited focus mainly on the presence of emotional states in narratives without linking characters to these states. To remedy this limitation, we propose automatically identifying characters and their emotions using sequence-to-sequence language models based on the transformer architecture.


\end{frame}

\begin{frame}{Existing research on computational dream analysis}


  %\citet{bertolini2023automatic} are the first to fine-tune a pre-trained language model for detecting the presence of emotions

  %\vspace{0.5cm}
  %\pause

  \begin{figure}
    \centering
    \includegraphics[width=0.8\linewidth]{img/bertolini_dream.png}
    \caption{\textbf{Architecture for multi-label emotion detection} \smallcitep{bertolini2023automatic}.}
    \label{fig:placeholder}
  \end{figure}

  %\vspace{0.25cm}
  \pause

  They use full context and compare predictions with gold annotations

  \vspace{0.25cm}
  \pause

  \textbf{Limitations}: emotions without characters; frequency not captured

  \vspace{0.25cm}
  \pause
  
  $\rightarrow$ We address this by identifying characters and their emotions with transformer-based text-to-text models

\end{frame}

%\begin{frame}{Encoder-decoder transformer architecture}
%\begin{figure}
%    \centering
%    \includegraphics[width=0.45\linewidth]{img/seq2seq.png}
    %\caption{\textbf{Sequence-to-sequence architecture based on transformers} \smallcitep{vaswani2017attention}. The encoder processes the input sequence, and the decoder generates the output sequence.}
%    \label{fig:placeholder}
%\end{figure}

%\smallcitet{vaswani2017attention}

%\end{frame}

\begin{frame}{T5 language models}
\begin{figure}
    \centering
    \includegraphics[width=1\linewidth]{img/t5_architecture.png}
    \caption{\textbf{Text-to-text framework} \smallcitep{2020t5}. One model maps input text to target text for tasks such as translation, QA, and classification.}
    \label{fig:placeholder}
\end{figure}

%\vspace{0.5cm}
\pause

$\rightarrow$ 248M-parameter LaMini-Flan-T5, pre-trained on 2.58M instructions across 15 tasks \smallcitep{lamini-lm}%, which takes token sequences as input and generates token sequences as output, and was among the top sub-1B models across 15 NLP tasks \parencite{lamini-lm}.

\end{frame}

\begin{frame}{Character and emotion detection in dream narratives}

  %[add list of characters and emotions]

%\textbf{Goal}: Character and emotion detection in dream narratives

%\pause

\begin{figure}
    \centering
    \includegraphics[width=0.8\linewidth]{img/dream_method.png}
    %\caption{Codes describing characters and their emotions are converted into
%natural language to produce the training data.}
    \label{fig:placeholder}
\end{figure}

\vspace{0.25cm}
\pause

$\rightarrow$ Our framework can be extended to include other HVdC categories %(social interactions, settings, objects)

%\scriptsize

%\textbf{G. Cortal}. \href{https://aclanthology.org/2024.lrec-main.1282/}{Sequence-to-Sequence Language Models for Character and Emotion Detection in Dream Narratives}. \textit{LREC-COLING 2024}.
    
\end{frame}

\begin{frame}{Results}

\texttt{Baseline} is LaMini-Flan-T5 finetuned on 1823 dream narratives

\begin{table}
  \centering
  % 1. Use font size and tabcolsep instead of resizebox to ensure
  % columns stay aligned across slides.
  \small
  \setlength{\tabcolsep}{5pt} 

    \caption{\textbf{Character and emotion detection}. \scriptsize{\textcolor{bluepanam}{$p<0.05$}.}}
  \label{tab:result_ablation}

  \vspace{-0.5cm}
  
  \begin{tabular}{lcccccc}
    \textbf{Model} & \textbf{Status} & \textbf{Gender} & \textbf{Identity} & \textbf{Age} & \textbf{Character} & \textbf{Emotion} \\
    \hline
    \texttt{Baseline} & 82.9 & 78.0 & 76.2 & 86.2 & 64.7 &  75.1 \\
    \hline
    
    % 2. Use \noalign{\pause} to prevent the blank row issue
    \pause
    
    \texttt{No\textsubscript{semantics}} & 71.4 & \textcolor{bluepanam}{56.5} & 61.0  & 90.5  & \textcolor{bluepanam}{41.8} &  75.8 \\
    \texttt{No\textsubscript{names}} & \textcolor{bluepanam}{80.7}  & \textcolor{bluepanam}{74.3}  & 74.2  & \textcolor{bluepanam}{84.0}  & \textcolor{bluepanam}{60.9} &  \textcolor{bluepanam}{73.0} \\
    \hline
    
    \pause
    
    \texttt{Size\textsubscript{small}} & \textcolor{bluepanam}{78.4}  & \textcolor{bluepanam}{72.1}  & \textcolor{bluepanam}{70.3}  & \textcolor{bluepanam}{81.7}  & \textcolor{bluepanam}{56.8} &  \textcolor{bluepanam}{70.2} \\
    \texttt{Size\textsubscript{large}} & \textcolor{bluepanam}{84.5} & \textcolor{bluepanam}{80.3}  & \textcolor{bluepanam}{78.6}  & 87.3  & \textcolor{bluepanam}{67.6} & 74.7 \\
    \hline
    
    \pause
    
    \texttt{First\textsubscript{group}} & 82.3  & 77.7  & 74.9  & 85.6  & 63.7 &  71.9 \\
    \texttt{First\textsubscript{individual}} & \textcolor{bluepanam}{80.6}  & 76.1  & \textcolor{bluepanam}{74.2}  & \textcolor{bluepanam}{83.9} & 62.7 &  67.3 \\
    \texttt{First\textsubscript{emotion}} & 83.9 & 78.7 & 77.1  & 87.6  & 65.0 &  72.0 \\
    \hline
    
    \pause
    
    \texttt{Conversion\textsubscript{comma}} & \textcolor{bluepanam}{84.0}  & \textcolor{bluepanam}{79.8}  & \textcolor{bluepanam}{77.7}  & \textcolor{bluepanam}{87.1}  & \textcolor{bluepanam}{66.7} &  73.7 \\
    \texttt{Conversion\textsubscript{marker}} & 82.4  & 78.5  & 76.5  & 86.1  & 65.4 &  74.4 \\
  \end{tabular}
\end{table}

%\vspace{0.25cm}

% The table sequence ends on slide 7, so we force this to slide 8

\onslide<6->{
$\rightarrow$ Our models can address this task; there is room for improvement

%\vspace{0.25cm}

{\scriptsize 58 $F_1$-score for gender prediction using lexical approaches \smallcitep{ourdreams}}

\vspace{-0.15cm}

{\scriptsize 86 $F_1$-score for emotion presence detection using transformers \smallcitep{bertolini2023automatic}}

}

    
\end{frame}

\begin{frame}{Case study on the war veteran}
  \begin{table}
    \centering
    \begin{tabular}{llccc}
%\toprule
\textbf{Group} & \textbf{Category} & \textbf{\% Vet} & \textbf{\% Total} & \textbf{$\Delta$} \\
\midrule
\multirow{5}{*}{Identity} & known & 24.9 & 51.6 & \textcolor{bluepanam}{-26.7} \\
& prominent & 1.9 & 2.5 & -0.6 \\
& occupational & 22.4 & 8.0 & \textcolor{bluepanam}{14.4} \\
& ethnic & 4.1 & 0.9 & \textcolor{bluepanam}{3.1} \\
& unknown & 46.8 & 37.0 & \textcolor{bluepanam}{9.8} \\
\addlinespace
%\pause
\multirow{4}{*}{Gender} & male & 56.2 & 43.0 & \textcolor{bluepanam}{13.1} \\
& female & 24.1 & 33.1 & \textcolor{bluepanam}{-9.0} \\
& joint & 10.9 & 12.2 & -1.3 \\
& undefined & 7.9 & 8.7 & -0.9 \\
%\hline
\end{tabular}
\caption{\textbf{Identity and gender proportions for the veteran} (n=566 narratives) versus other dreamers. \scriptsize{$\Delta$ shows the difference in percentage points; \textcolor{bluepanam}{$p<0.05$}.}}
\label{tab:case_study_war_veteran}
\end{table}

\vspace{-0.5cm}

\onslide<2->{
$\rightarrow$ The veteran dreams more about \textit{occupational}, \textit{ethnic}, and \textit{unknown} identities compared to other dreamers}

\vspace{0.25cm}

\onslide<2->{\footnotesize{Generated annotations for DreamBank are available on \href{https://huggingface.co/gustavecortal}{hf.co/gustavecortal}}}

\end{frame}

\section{Formalization of style}

\begin{frame}{}
\Large
\vfill
\begin{center}
    Formalization of style in personal narratives
    %\section{Formalization of style in personal narratives}
\end{center}

\vfill

\footnotesize

\textbf{G. Cortal} and A. Finkel. \href{https://gustavecortal.com/data/Formalizing_Style_in_Personal_Narratives.pdf}{Formalizing Style in Personal Narratives}. \textit{EMNLP 2025}.
\end{frame}

\begin{frame}{Motivation}

  \textbf{Limitation}: A formalization of style that captures how subjective experience is linguistically communicated is lacking

  \vspace{0.5cm}
  \pause

  Formalization could enable more precise identification of linguistic patterns associated with psychological states and may support interventions \smallcitep{whiteNarrativeMeansTherapeutic1990} 

  \vspace{0.5cm}
  \pause

  %$\rightarrow$ Our approach complements existing works by providing a formal framework grounded in systemic functional linguistics %\smallcitep{hallidayIntroductionFunctionalGrammar2014}

  %\vspace{0.5cm}
  %\pause

  $\rightarrow$ We aim to create an accessible, formal framework that researchers can build upon in future studies % We do not claim to capture the full complexity of subjective experience in text

\end{frame}

\begin{frame}{Motivation}

  Scholarly work has examined personal modes of reasoning and expression \smallcitep{hadamard1945essay,granger1968essai,husserl2012ideas,dilts1994strategies}

    \vspace{0.5cm}
    \pause

    $\rightarrow$ They describe "styles of thought" but lack operational tools

    \vspace{0.5cm}
    \pause

    Style is central to how authors express themselves: stylistics \smallcitep{walesDictionaryStylistics2014a}, stylometry \smallcitep{nealSurveyingStylometryTechniques2017} 

  \vspace{0.5cm}
    \pause

    In NLP, style transfer aims to control linguistic attributes while preserving semantic content \smallcitep{jinDeepLearningText2022,troianoTheoriesStylesTheir2023a}

        \vspace{0.5cm}
    \pause

  $\rightarrow$ They provide operational tools to capture or control linguistic form, but do not focus on how such forms encode subjective experience

\end{frame}

%\begin{frame}{Motivation}

%  \citet{TellierFinkel95} define linguistic style as lexical and syntactical patterns using formal language theory %[be more precise]

%  \vspace{0.5cm}
%  \pause

%  \citet{pachet} use constrained Markov processes to generate lyrics that match an author's style while satisfying rhyme and meter constraints %style as markov chains, short-term probabilities, music domain

%  \vspace{0.5cm}
%  \pause
  
%  $\rightarrow$ We retain the idea 


%\end{frame}

\begin{frame}{How to give an operational definition of style?}

\textbf{Intuitive definition}: a distinctive manner of communicating subjective experience in personal narratives

\vspace{0.5cm}
\pause

\textbf{Hypothesis}: An author uses some redundant choices of features that characterize its style

\vspace{0.5cm}
\pause

\textbf{Goal}: Map narratives to sequences based on extracted linguistic features:

“I wake in a dark room. I feel a cold wind. I tell myself to move.” $\rightarrow$ $amv$

%\textbf{Research task}: Formalize style as \textit{patterns of linguistic choices that encode subjective experience}

%\vspace{0.5cm}
%\pause
    
\end{frame}

\begin{frame}{Contributions}

\begin{itemize}[<+->]
    \item A sequence-based framework defining style as \textit{patterns in sequences of linguistic choices that encode subjective experience}% grounded in systemic functional linguistics
    \item A methodology for identifying patterns using sequence analysis
    \item A case study on dream narratives %, showing how the analysis of patterns can reveal psychological insights
\end{itemize}
    
\end{frame}

\begin{frame}{What linguistic features encode subjective experience?}

\begin{columns}[T,onlytextwidth]
  \begin{column}{0.7\textwidth}
    % 1. Wrap the first paragraph to appear on Slide 1+
    \onslide<1->{
        We ground our framework in \textit{systemic functional linguistics} \smallcitep{hallidayIntroductionFunctionalGrammar2014a}
    }

    \vspace{0.5cm}

    % 2. Wrap the second paragraph to appear on Slide 2+
    \onslide<2->{
        Meaning emerges through choices in systems of linguistic features to achieve communicative goals
    }

    \vspace{0.5cm}

    % 3. Wrap the third paragraph to appear on Slide 3+
    \onslide<3->{
        Language achieves three functions:
        \begin{itemize}
            \item Interpersonal: language builds social relationships
            \item Textual: information is organized to create coherent messages
            \item \textit{Ideational}: language represents experience through processes and participants
        \end{itemize}
    }
  \end{column}

  \begin{column}{0.25\textwidth}
    % Because we removed \pause, this column is now visible on all slides (1-3) by default.
    \begin{figure}
      \centering
      \includegraphics[width=\linewidth]{img/halliday_book_cover_upscale.jpg}
      \label{fig:placeholder}
      \caption{\citet{hallidayIntroductionFunctionalGrammar2014a}. $+57,000$ citations.}
    \end{figure}
  \end{column}
\end{columns}
 
\end{frame}

\begin{frame}{What linguistic features encode subjective experience?}

  According to the \textit{ideational function}, language represents experience through \textcolor{DarkBlue}{\textbf{processes}} and \textcolor{blueNCS}{participants}

  %\vspace{0.05cm}
  %\pause

\begin{table}[!htb]
  \centering
  \renewcommand{\arraystretch}{2.3}
  \resizebox{\textwidth}{!}{
  \begin{tabular}{p{4.7cm}|p{7.3cm}}
    %\hline
    \textbf{Processes} & \textbf{Examples} \\ \hline
    \texttt{Action}: actions and events in the physical world. &
    \textcolor{blueNCS}{He}\textsubscript{\texttt{Actor}} \textcolor{DarkBlue}{\textbf{takes}}\textsubscript{\texttt{Action}} \textcolor{blueNCS}{the valuable}\textsubscript{\texttt{Affected}} \newline
    
    %\textcolor{blueNCS}{Members of my cult}\textsubscript{\texttt{Actor}} \textcolor{DarkBlue}{\textbf{have made}}\textsubscript{\texttt{Action}} \textcolor{blueNCS}{1500 euros}\textsubscript{\texttt{Result}} \newline
    
    \textcolor{blueNCS}{I}\textsubscript{\texttt{Actor}} \textcolor{DarkBlue}{\textbf{give}}\textsubscript{\texttt{Action}} \textcolor{blueNCS}{her}\textsubscript{\texttt{Recipient}} \textcolor{blueNCS}{a chance}\textsubscript{\texttt{Range}} \\ \hline
    
    \texttt{Mental}: internal experiences such as thoughts, perceptions, and feelings. &
    %\textcolor{blueNCS}{We}\textsubscript{\texttt{Senser}} \textcolor{DarkBlue}{\textbf{believe}}\textsubscript{\texttt{Mental}} \textcolor{blueNCS}{women are the leaders of change}\textsubscript{\texttt{Phenomenon}} \newline
    
    \textcolor{blueNCS}{The moon}\textsubscript{\texttt{Senser}} \textcolor{DarkBlue}{\textbf{sees}}\textsubscript{\texttt{Mental}} \textcolor{blueNCS}{the earth}\textsubscript{\texttt{Phenomenon}} \newline
    
    \textcolor{blueNCS}{He}\textsubscript{\texttt{Senser}} \textcolor{DarkBlue}{\textbf{disliked}}\textsubscript{\texttt{Mental}} \textcolor{blueNCS}{Gilbert's writing}\textsubscript{\texttt{Phenomenon}} \\ \hline
    
    \texttt{Verbal}: acts of communication. &
    \textcolor{blueNCS}{David}\textsubscript{\texttt{Sayer}} \textcolor{DarkBlue}{\textbf{said}}\textsubscript{\texttt{Verbal}} \textcolor{blueNCS}{``the corrupt, [...]''}\textsubscript{\texttt{Verbiage}} \\ \hline

    \texttt{State}: states of being, having, or existence. &
     %\textcolor{DarkBlue}{\textbf{was}}\textsubscript{\texttt{Existential}} \textcolor{blueNCS}{a swimming pool}\textsubscript{\texttt{Existent}} \newline
    
    \textcolor{blueNCS}{Clément}\textsubscript{\texttt{Carrier}} \textcolor{DarkBlue}{\textbf{is}}\textsubscript{\texttt{State}} \textcolor{blueNCS}{a teacher}\textsubscript{\texttt{Attribute}} \newline
    
    \textcolor{blueNCS}{Arthur}\textsubscript{\texttt{Possessor}} \textcolor{DarkBlue}{\textbf{has}}\textsubscript{\texttt{State}} \textcolor{blueNCS}{a cat}\textsubscript{\texttt{Possessed}} \\ %\hline
  \end{tabular}}
  \label{tab:process_participants}
  %\caption{According to \textit{systemic functional linguistics}, language represents experience through \textcolor{DarkBlue}{\textbf{processes}} and \textcolor{blueNCS}{participants}.}% and \textit{circumstances}}
\end{table}

%\vspace{0.15cm}
%\footnotesize
%\textbf{Legend:} \textcolor{DarkBlue}{\textbf{Process}} (bold, dark blue) \quad
%\textcolor{blueNCS}{Participant} (blue) \quad
%\textit{Role labels} in subscripts (e.g., \texttt{Actor}, \texttt{Senser})
    
%[make it more visual, less text-heavy, maybe with colors]

\end{frame}

\begin{frame}{Formal definition of style}

\textbf{Alphabet}: Let $\Sigma$ be the set of process types
$$ \Sigma = \{ \texttt{Action}, \texttt{Mental}, \texttt{Verbal}, \texttt{State}\} $$

\vspace{0.2cm}
\pause

\textbf{Text}: Let $\mathcal{C}$ be the set of all natural language clauses. We define a text $T$ as a finite sequence of clauses:
$$ T = (c_1, c_2, \dots, c_n) \in \mathcal{C}^n $$

\vspace{0.2cm}
\pause

\textbf{Mapping}: Each text $T$ is mapped to a sequence $\phi(T)$ over the alphabet $\Sigma$. Let $\phi: \mathcal{C}^* \to \Sigma^*$ be a function mapping clauses to symbol sequences:
$$ \phi(T) = (y_1, y_2, \dots, y_n) \in \Sigma^n $$
%\vspace{-0.1cm}
%\centering
%\small
%where $y_i \in \Sigma$ is the letter assigned to clause $c_i$% by the model

\vspace{0.3cm}
\pause
\raggedright
\normalsize

$\rightarrow$ We implement $\phi$ using a language model with in-context learning

\end{frame}


\begin{frame}{Formal definition style}

\textbf{Style of text}: We define the style of text $T$ as the set of patterns contained in its sequence $\phi(T)$
$$ \mathcal{S}(T) = \{ w \in \Sigma^* \mid w \subseteq \phi(T) \} $$

\vspace{0.1cm}
\centering
\small
where $w \subseteq \phi(T)$ denotes a substring (contiguous symbols)

\vspace{0.5cm}
\pause
\normalsize
\raggedright

\textbf{Author corpus}: Let $\mathcal{C}_A = \{T_1, T_2, \dots, T_k\}$ be the finite set of narrative texts attributed to author $A$

\vspace{0.5cm}
\pause

\textbf{Style of author}: The style of author $A$, denoted $\mathcal{S}(A)$, is the union of patterns contained in each text $T \in \mathcal{C}_A$

\vspace{0.15cm}

$$ \mathcal{S}(A) = \bigcup_{T \in \mathcal{C}_A} \mathcal{S}(T) $$

\end{frame}

%\textit{We implement $\phi$ using a language model with in-context learning}

\begin{frame}{Methodology for our sequence-based framework}

%\textbf{Goal}: Map narratives to sequences based on extracted linguistic features:

%\vspace{0.25cm}
%\pause

\textbf{Narrative}: \small{“I wake in a dark room. I feel a cold wind. I tell myself to move.”} %$\rightarrow$ $amv$

%\pause

%\vspace{0.5cm}


\begin{table}[!ht]
  \centering
  %\small
  \renewcommand{\arraystretch}{1.8}
  \begin{threeparttable}
    %\caption{Illustrative pipeline for our sequence-based framework. We first segment “I wake in a dark room. I feel a cold wind. I tell myself to move.” into clauses, then identify features such as processes and participants for each clause. Each text is mapped to a symbolic sequence using an alphabet based on extracted features.}
    \label{tab:example}
    \begin{tabular}{lll}
      %\toprule
      \textbf{Clause} & \textbf{Process (symbol)} & \textbf{Participants} \\
      \midrule
      I wake in a dark room         & \texttt{Action} (\textbf{a})  & \texttt{Actor} \\
      I feel a cold wind            & \texttt{Mental} (\textbf{m})  & \texttt{Senser},\\
                                            &             & \texttt{Phenomenon} \\
      I tell myself to move         & \texttt{Verbal} (\textbf{v})  & \texttt{Sayer},\\
                                            &             & \texttt{Recipient} \\
      \bottomrule
    \end{tabular}

    \begin{tablenotes}[flushleft]
      %\large
      \item \textbf{Sequence:} $amv$\quad|\quad
            \textbf{Substrings:} \{am, mv\}
    \end{tablenotes}
  \end{threeparttable}
\end{table}

%\vspace{0.25cm}

%\small
%\textit{Substrings are contiguous sequences of symbols within a sequence}

%\begin{enumerate}[<+->]
%    \item We first segment \textit{“I wake in a dark room. I feel a cold wind. I tell myself to move.”} into clauses
%    \item Identify features (\textit{e.g.}, processes and participants) for each clause using in-context learning with large language models
%    \item Each narrative is mapped to a symbolic sequence using an alphabet based on identified features
%    \item We perform sequence analysis to identify patterns such as frequent substrings and representative sequences
%\end{enumerate}
    
\end{frame}

\begin{frame}{Results on the war veteran}

\begin{figure}
    \centering
    \includegraphics[width=0.7\linewidth]{img/dendogram_viet.png}
    \caption{Dendrogram with Ward linkage and cosine similarity}% on substrings of size one, two, and three.}
    \label{fig:dendogram}
\end{figure}

%\vspace{0.25cm}
\pause

{%\small
\textbf{Representative sequences}: \textit{savamasasaaamaaasavvvaaaaaaavssaaaaa} and \textit{sssssavaavssvsavvvvsmasasaasasaamaamvmsss} 

\small with $a=\text{action},\; m=\text{mental},\; s=\text{state},\; v=\text{verbal}$
}

%\vspace{0.25cm}
%\pause

%$\rightarrow$ Two templates: a highly action-oriented structure or a more varied structure alternating between state and action processes
    
\end{frame}

\begin{frame}{Results on the war veteran}

We compare the proportion of sequences containing a given substring

\pause

\begin{figure}[!htb]
     \begin{subfigure}[b]{0.5\textwidth}
         \centering
         \includegraphics[scale=0.2]{img/viet_odds_2.png}
         \caption{Size 2.}
         \label{fig:viet_odds2}
     \end{subfigure}
         \begin{subfigure}[b]{0.4\textwidth}
         \centering
         \includegraphics[scale=0.2]{img/viet_odds_3.png}
         \caption{Size 3.}
         \label{fig:viet_odds3}
     \end{subfigure}
        \caption{Top substring odds ratio between the veteran and the norm}
        \label{fig:viet_odds}
\end{figure}

\pause


$\rightarrow$ The veteran favors verbal processes over mental ones% with high odds ratios (respectively 2.00 and 1.75)

\pause


$\rightarrow$ Our results can inform psychological interpretations; need more individuals to generalize findings

\end{frame}

\begin{frame}{How can this framework be extended?}

  %Retour à l'ingénierie des traits: utilisation de la puissance des modèles de langues pour extraire des traits linguistiques, que l'on analyse ensuite avec des méthodes d'analyse de séquence 


\begin{itemize}[<+->]
    \item \textbf{Incorporating additional linguistic features}: duration of processes, concreteness of participants %\smallcitep{hallidayIntroductionFunctionalGrammar2014a,troianoTheoriesStylesTheir2023a}%: adding features related to the interpersonal and textual functions of language
    \item \textbf{Authorship profiling}: classifying authors based on their sequences \smallcitep{ferraraDetectionPromotedSocial2016}
    %[map sequences to authors]
    %: identifying signature patterns (\textit{e.g.}, distinctive substrings) that characterize an author's unique way of constructing narratives
    \item \textbf{Style-conditioned narrative generation} \smallcitep{pachet,alhafniPersonalizedTextGeneration2024b}%: generating narratives from a sequence of linguistic features
    \item \textbf{Applying methods from complexity science} \smallcitep{lempelComplexityFiniteSequences1976,hipolitoPatternBreakingComplex2023a}% and formal language theory%: analyzing subsequences, using complexity measures to quantify redundancies, etc.
\end{itemize}

\end{frame}

\section{Conclusion and perspectives}

\begin{frame}{}
\Large
\vfill
\begin{center}
    Conclusion and perspectives
    %\section{Thematic analysis in mental health narratives using language models}
\end{center}
\vfill

\end{frame}

\begin{frame}{Conclusion}

How to model subjective experience in personal narratives?

\vspace{0.5cm}
\pause

%\begin{itemize}[<+->]
%    \item Cognitive science perspective on emotion analysis
%    \item French narratives based on emotion components
%    \item Emotion analysis in emotional and dream narratives 
%    \item Formalization of style in personal narratives
%\end{itemize}

\begin{itemize}[<+->]
  \setlength{\itemsep}{1.0em}
  \item Cognitive science perspective on emotion analysis
  \begin{itemize}[<+->]\footnotesize
    \setlength{\itemsep}{0.5em}
    \item Overview of psychological theories with emotion annotation schemes
    \item Limitations and research directions for emotion analysis
  \end{itemize}
  \item New \href{https://huggingface.co/datasets/gustavecortal/FrenchEmotionalNarratives}{French corpus of narratives based on emotion components}
  \item Emotion analysis in emotional and dream narratives 
  \begin{itemize}[<+->]\footnotesize
    \setlength{\itemsep}{0.5em}
    \item First \href{https://huggingface.co/gustavecortal/distilcamembert-cae-all}{language model for emotion prediction based on components}
    \item First \href{https://huggingface.co/gustavecortal/dream-t5}{language model for character and emotion prediction in dreams}
  \end{itemize}
  \item Formalization of style in personal narratives
\end{itemize}

\vspace{0.5cm}

\onslide<1->{
{\small
\textbf{Papers}: 2 int. conferences, 3 int. workshops, 2 national venues}}

%\vspace{0.25cm}

%\textbf{Open corpus and tools}: \href{https://huggingface.co/datasets/gustavecortal/FrenchEmotionalNarratives}{French corpus based on emotion components}; language models for emotion analysis in \href{https://huggingface.co/gustavecortal/distilcamembert-cae-all}{emotional} and \href{https://huggingface.co/gustavecortal/dream-t5}{dream} narratives
%}}


\end{frame}

\begin{frame}{Perspectives}

  \begin{itemize}[<+->]
    \item \textbf{Emotion analysis for mental health}: empathic support, cognitive distortions, theory of mind \smallcitep{gandhiUnderstandingSocialReasoning2023,maHolisticLandscapeSituated2023,sharmaCognitiveReframingNegative2023b}
    \item \textbf{Post-training for psychology}: preferences and reasoning data \smallcitep{zhang2025preferencelearningunlocksllms}
    \item \textbf{Psychology of language models}: sycophancy, thought operations \smallcitep{didolkar2025metacognitivereuseturningrecurring,sharma2025understandingsycophancylanguagemodels}
  \end{itemize}

%\vspace{0.5cm}
%\pause

%I co-supervised ongoing PhD thesis: Nicolas Richet (multimodal emotion recognition), Amine Haddou (cognitive distortions), and Raphael Faure (style analysis)
  
\end{frame}

\section{Impact}

\begin{frame}{}
\Large
\vfill
\begin{center}
    Impact
    %\section{Thematic analysis in mental health narratives using language models}
\end{center}
\vfill

\end{frame}

\begin{frame}
    \frametitle{Ongoing PhD thesis related to my works}

\begin{figure}[!htb]
  \centering
  \begin{subfigure}[b]{0.32\textwidth}
    \centering
    \includegraphics[width=\linewidth]{img/face/face_amine.png}
    \caption{A. Haddou on cognitive distortions\\(2025, ENS Paris-Saclay).}
    \label{fig:face_amine}
  \end{subfigure}\hfill
  \begin{subfigure}[b]{0.32\textwidth}
    \centering
    \includegraphics[width=\linewidth]{img/face/face_raphael.jpeg}
    \caption{R. Faure on style analysis\\(2025, ENS Paris-Saclay).}
    \label{fig:face_raphael}
  \end{subfigure}\hfill
  \begin{subfigure}[b]{0.32\textwidth}
    \centering
    \includegraphics[width=\linewidth]{img/face/face_nicolas.jpeg}
    \caption{N. Richet on multimodal emotion\\ (2024, ETS Montréal).}
    \label{fig:face_nicolas}
  \end{subfigure}
  %\caption{Ongoing PhD thesis related to my works.}
  \label{fig:faces}
\end{figure}

\end{frame}

\begin{frame}
    \frametitle{NLP for psychiatry (industry)}

I wanted to apply my NLP skills to industry work with social impact

\vspace{1cm}
\pause

\noindent
\begin{minipage}[c]{0.7\textwidth}
    6-month PhD internship at Callyope on \textit{NLP for quantifying memory, future thinking, and the self in mental health narratives}
\end{minipage}%
\hfill
\begin{minipage}[c]{0.25\textwidth}
    \centering
    \includegraphics[scale=0.12]{img/logo_callyope.png}
\end{minipage}


\end{frame}

\section{Automatic thematic analysis in mental health narratives}

\begin{frame}{}
\Large
\vfill
\begin{center}
    Automatic thematic analysis in mental health narratives using language models% (change title to thematic analysis?)
    %\section{Thematic analysis in mental health narratives using language models}
\end{center}

\vfill

\footnotesize

\textbf{G. Cortal}, S. Guessoum, X. Cao, R. Riad. \textit{Fine-grained mental health topic modeling in different cohorts using large language models} (preprint). 2025. 

\end{frame}

\begin{frame}{Motivation}

  \begin{itemize}[<+->]
    \item Qualitative analysis of speech content is central to clinical practice
    \item Thematic analysis studies how people construct meaning% from their experiences
    \item Thematic analysis is time-consuming, often constrained to small, monolingual corpora
    \item Computational approaches offers time savings, can analyze a larger amount of data
  \end{itemize}

  %\pause
  %\vspace{0.5cm}

  %$\rightarrow$ We developed a pipeline that (a) clusters narratives from different cohorts, (b) generates descriptions for each cluster, and (c) links clusters to variation in clinical scores and sociodemographic factors %We then quantified which questions best separate clinically relevant clusters, and we conducted an analysis of confounding by age, education, and sex, an aspect often overlooked in topic modeling despite strong demographic effects on language use.

\end{frame}
  

\begin{frame}{Data collection}

Narratives and clinical scores from \textbf{four cohorts}: French general population (n=1809) and three clinical cohorts (Italian n=116, Chinese n=52, Spanish n=90)

\vspace{0.5cm}
\pause

\textbf{Clinical scores} for depression, anxiety, insomnia

\vspace{0.5cm}
\pause

\textbf{Open-ended questions} involving last 24h, positive future event, current feelings and sleep, etc.

\end{frame}

\begin{frame}{Semantic clustering and description generation}

  %We developed a multilingual pipeline that (a) clusters spontaneous speech transcripts from four cohorts (general population and three clinical samples in French, Italian, Chinese, and Spanish), (b) generates fine-grained natural-language descriptions for each cluster, and (c) links clusters to variation in clinical scores and sociodemographic factors. We then quantified which questions best separate clinically relevant clusters, and we conducted an analysis of confounding by age, education, and sex, an aspect often overlooked in topic modeling despite strong demographic effects on language use. 

\begin{figure}
    \centering
    \includegraphics[scale=0.6]{img/topic_modeling/methods_v3.png}
    %\caption{\textbf{Computational pipeline for semantic clustering and description generation}. Speech transcripts are converted to semantic vectors via multilingual embeddings, dimensionally reduced, and grouped into clusters using density-based methods. Each cluster undergoes: (1) statistical analysis linking cluster membership to clinical/demographic  variables, and (2) automatic description generation where a large language model summarizes transcripts per cluster into human-readable natural language. This dual quantitative-qualitative output enables both statistical hypothesis testing and clinical interpretation of discovered topics.}
    %\caption{Each cluster undergoes: (1) statistical analysis linking cluster membership to clinical/demographic  variables, and (2) automatic description generation where a large language model summarizes transcripts per cluster into human-readable natural language.}
    \label{fig:methods}
\end{figure}

\end{frame}

\begin{frame}{Distribution of depression scores across clusters}

  \textit{Current feelings and sleep} (n=1,786)
  \begin{figure}
    \centering
    \includegraphics[scale=0.06]{img/topic_modeling/boxplot_description/popgen_description_larger_boxplots.png}
    %\caption{Distribution of PHQ-9 depression scores across automatically identified clusters in the French general population cohort (n=1,786 transcripts). Responses to \textit{Describe how you are feeling at the moment and how your sleep has been lately}.}
    %\caption{Distribution of PHQ-9 depression scores across automatically identified clusters in the French general population cohort (n=1,786 transcripts). Responses to the prompt 'Describe how you are feeling at the moment and how your sleep has been lately'. Box plots show median (center line), interquartile range (box), and 1.5x IQR whiskers. Effect size (Kruskal-Wallis H = 0.17, p < 0.00001) indicates moderate discrimination between clusters. Sample sizes per cluster range from n=34 to n=92. Selected cluster descriptions (Clusters 1, 10, 12, 26) were generated by Qwen3-14B summarizing 30 random transcripts per cluster.}
    \label{fig:popgen_description}
\end{figure}


%\vspace{0.1cm}
\pause

$\rightarrow$ Depression scores vary significantly: cluster 26 highest (13.4±5.4), cluster 1 lowest (2.6±2.2)


%For the French general population cohort based on answers to “Describe how you’re feeling and how your nights have been”, among 26 identified clusters (n=1,786 transcripts), PHQ-9 scores differed significantly across clusters $(H=0.17, p\leq0.00001)$. Cluster 26 (n=37, age=25±9) exhibited the highest depression scores PHQ-9 (13.4±5.4) with LLM-generated descriptions highlighting \textit{sleep disturbances characterized by insomnia, frequent awakenings, and restless sleep}, alongside pervasive \textit{anxiety, emotional instability, and self-esteem issues}. In contrast, Cluster 1 (n=35, age=39±19) showed the lowest scores PHQ-9 (2.6±2.2), with descriptions emphasizing\textit{ consistent satisfaction with current well-being, good sleep quality, and general relaxation}.

%Distribution of PHQ-9 depression scores across automatically identified clusters in the French general population cohort
\end{frame}

\begin{frame}{Generated cluster descriptions}
  \begin{figure}
    \centering
    \includegraphics[scale=0.25]{img/topic_modeling/boxplot_description/popgen_description_larger_descriptions.png}
    %\caption{Distribution of PHQ-9 depression scores across automatically identified clusters in the French general population cohort (n=1,786 transcripts). Responses to the prompt 'Describe how you are feeling at the moment and how your sleep has been lately'. Box plots show median (center line), interquartile range (box), and 1.5x IQR whiskers. Effect size (Kruskal-Wallis H = 0.17, p < 0.00001) indicates moderate discrimination between clusters. Sample sizes per cluster range from n=34 to n=92. Selected cluster descriptions (Clusters 1, 10, 12, 26) were generated by Qwen3-14B summarizing 30 random transcripts per cluster.}
    \label{fig:popgen_description}
\end{figure}

\pause

$\rightarrow$ Clustering captures symptom severity and age-related circumstances

%Selected cluster descriptions (Clusters 1, 10, 12, 26) were generated by Qwen3-14B summarizing 30 random transcripts per cluster.
\end{frame}

%\section{Perspectives}

%\begin{frame}{}
%\Large
%\vfill
%\begin{center}
%    Perspectives
    %\section{Thematic analysis in mental health narratives using language models}
%\end{center}

%\vfill

%\end{frame}

\begin{frame}{Conclusion}

How to model subjective experience in personal narratives?

\vspace{0.25cm}
\begin{itemize}
  \setlength{\itemsep}{1.0em}
  \item Cognitive science perspective on emotion analysis
  \begin{itemize}\footnotesize
    \setlength{\itemsep}{0.5em}
    \item Overview of psychological theories with emotion annotation schemes
    \item Limitations and research directions for emotion analysis
  \end{itemize}
  \item New \href{https://huggingface.co/datasets/gustavecortal/FrenchEmotionalNarratives}{French corpus of narratives based on emotion components}
  \item Emotion analysis in emotional and dream narratives 
  \begin{itemize}\footnotesize
    \setlength{\itemsep}{0.5em}
    \item First \href{https://huggingface.co/gustavecortal/distilcamembert-cae-all}{language model for emotion prediction based on components}
    \item First \href{https://huggingface.co/gustavecortal/dream-t5}{language model for character and emotion prediction in dreams}
  \end{itemize}
  \item Formalization of style in personal narratives
\end{itemize}

\vspace{0.5cm}


{\small
\textbf{Papers}: 2 int. conferences, 3 int. workshops, 2 national venues

%\vspace{0.25cm}

%\textbf{Open corpus and tools}: \href{https://huggingface.co/datasets/gustavecortal/FrenchEmotionalNarratives}{French corpus based on emotion components}; language models for emotion analysis in \href{https://huggingface.co/gustavecortal/distilcamembert-cae-all}{emotional} and \href{https://huggingface.co/gustavecortal/dream-t5}{dream} narratives
}

\end{frame}

\appendix
\begin{frame}{Selected research papers}
  \scriptsize

    \fullcitenourl{bonardImprovingLanguageModels2024a}

    \vspace{0.25cm}

    \fullcitenourl{cortalEmotionRecognitionBased2023d}

    \vspace{0.25cm}

    \fullcitenourl{cortalSequencetoSequenceLanguageModels2024}

    \vspace{0.25cm}

    \fullcitenourl{cortal-finkel-2025-formalizing}
\end{frame}

\section{Appendix}

\begin{frame}{}
\Large
\vfill
\begin{center}
    Appendix
    %\section{Appendix}
\end{center}
\vfill

\end{frame}

\begin{frame}{What are the psychological mechanisms used to infer what is communicated?}

A \textit{code} is a pre-established pairing between stimuli and sets of information

\pause
\vspace{0.5cm}

The Morse code is a pairing between <combination of short and long signals> and [letters]% that senders and receivers must share to communicate with it. 

\pause
\vspace{0.5cm}

The formal semantics of a language is made of syntactical and lexical rules that pairs <strings of words> with [sentential meanings] %\cite{heim_semantics_1998}

%What are the psychological mechanisms used to infer what is communicated? %\cite{grice_logic_1975}


\end{frame}

\begin{frame}{What are the psychological mechanisms used to infer what is communicated?}

\begin{figure}
    \centering
    \includegraphics[scale=0.20]{img/scherer_dictionary_eng.png}
    \caption{Dictionary analysis in cognitive pragmatics.}% [cite]}
\end{figure}


\end{frame}

\begin{frame}{Codes underdetermine emotion meaning}

%\pause

Let's take emotion expression modes as an example:

\vspace{0.5cm}
\pause

\begin{itemize}[<+->]
  \item \textit{Labeled}: \enquote{I am happy now} is explicit about the feeling but does not encode what the emotion is about
  \item \textit{Displayed}: interjections (\enquote{Wow!}, \enquote{Ah!}, \enquote{Damn!}) show affect yet leave valence and focus unclear
  \item \textit{Suggested}: \enquote{The ship has black sails.} can communicate any kind of emotion
\end{itemize}

\vspace{0.5cm}
\pause


$\rightarrow$ We rely on other sources of evidence to infer what is communicated

\end{frame}

\begin{frame}{What are the psychological mechanisms used to infer what is communicated?}

%What are the psychological mechanisms used to infer what is communicated? %\cite{grice_logic_1975}

%\pause
%\vspace{0.5cm}

\begin{figure}
    \centering
    \includegraphics[scale=0.20]{img/scherer_dictionary_eng.png}
    \caption{Dictionary analysis in cognitive pragmatics.} %[cite]}
\end{figure}


\end{frame}

\begin{frame}{What are the psychological mechanisms used to infer what is communicated?}

%What are the psychological mechanisms used to infer what is communicated? %\cite{grice_logic_1975}

%\pause
%\vspace{0.5cm}

\begin{figure}
    \centering
    \includegraphics[scale=0.20]{img/scherer_eng.png}
    \caption{Detective analysis in cognitive pragmatics.} %[cite]}
\end{figure}


\end{frame}

\begin{frame}{Component classification in emotional narratives}

\begin{table}%%[!htb]
    \centering
\begin{tabular}{lllll}
\hline
\textbf{Model} &    \textbf{Precision} &      \textbf{Recall} &    \textbf{\boldmath{$F_1$}} \\
\hline
Logistic Regression &  84.9\,\mscriptsize{(0.3)} &  84.3\,\mscriptsize{(0.3)} &  84.4\,\mscriptsize{(0.3)} \\
CamemBERT & \textbf{93.2} &  \textbf{93.0} &  \textbf{93.1} \\
\hline
\end{tabular}
    \caption{Scores ($\pm$ std) for emotion component classification.}
    \label{tab:app_auto}
\end{table}

\pause

$\rightarrow$ Models can be used to automatically classify unstructured narratives

\end{frame}

\begin{frame}{Results}

\texttt{StableBeluga\textsubscript{i}} is a 7B model with in-context learning using $i$ examples

\begin{table}
    \centering
    \resizebox{1.0\textwidth}{!}{
\begin{tabular}{lcccccc}
\textbf{Model} & \textbf{Status} & \textbf{Gender} & \textbf{Identity} & \textbf{Age} & \textbf{Character} & \textbf{Emotion} \\
\hline
\texttt{Baseline} & 82.87 & 78.02 & 76.17 & 86.21 & 64.74 &  75.13 \\
\hline
\texttt{StableBeluga\textsubscript{1}} &   43.95\textbf{**} &   39.76\textbf{**} &     31.25\textbf{**} &  56.16\textbf{**} &      15.65\textbf{**} & -\\
\texttt{StableBeluga\textsubscript{3}} &   52.44\textbf{**} &   46.49\textbf{**} &     38.46\textbf{**} &  63.88\textbf{**} &      21.06\textbf{**} & -\\
\texttt{StableBeluga\textsubscript{5}} &   55.89\textbf{**} &   46.29\textbf{**} &     42.61\textbf{**} &  63.73\textbf{**} &      24.86\textbf{**} & - \\
\end{tabular}}
\caption{$F_1$-scores for character and emotion detection. Significant differences from \texttt{baseline}: ** ($p<0.01$), * ($p<0.05$).}
    %\caption{$F_1$-score of character and emotion detection in dream narratives. Significant differences from the baseline are indicated with ** ($p<0.01$) and * ($p<0.05$).}
    \label{tab:result}
\end{table}

\vspace{0.25cm}
\pause

$\rightarrow$ Compared to \texttt{StableBeluga}, our supervised models perform better while having 28 times fewer parameters (248M \textit{vs.} 7B)
    
\end{frame}

\begin{frame}{Effect size across questions and clinical scores}

\begin{figure}
    \centering
    \includegraphics[scale=0.35]{img/topic_modeling/heatmap_effect_sizes/V5_V6_V7_V8_V9_V10_phq9_gad7_bdi_ais_mfi_global_heatmap.png}
    %\caption{Effect size across questions and clinical scores for the French general population cohort}
    \label{fig:popgen_clinical_heatmap}
\end{figure}

\pause
$\rightarrow$ Certain questions better discriminate clinical scores

\end{frame}

\begin{frame}{Effect size across questions and sociodemographics}

\begin{figure}
  \centering
  \includegraphics[scale=0.35]{img/topic_modeling/heatmap_effect_sizes/V5_V6_V7_V8_V9_V10_diploma_level_gender_age_global_heatmap.png}
  %\caption{Effect size across questions and sociodemographics for the French general population cohort}
  \label{fig:popgen_demo_heatmap}
\end{figure}

\pause

$\rightarrow$ Nearly all questions discriminate sociodemographics

\end{frame}

\begin{frame}{Demographics}

\begin{table}[!htbp]
\centering
\resizebox{1\textwidth}{!}{
\begin{tabular}{p{4cm}llll}
 &
\textbf{\shortstack[l]{General\\Population}} &
\textbf{Androids} &
\textbf{MODMA} &
\textbf{VOCES} \\
 &
\textbf{n=1809} &
\textbf{n=116} &
\textbf{n=52} &
\textbf{n=90} \\

\hline
\textbf{Demographics} & & & & \\

\textbf{Language} & French & Italian & Chinese & Spanish \\

\textbf{Age} & *** & \textit{n.s.} & \textit{n.s.} & *** \\
Mean (SD) & 37.8 (18.2) & 37.4 (12.0) & 31.3 (9.2) & 38.6 (14.9) \\
Range     & 18--91       & 19--71      & 18--52      & 21--76       \\

\textbf{Sex, n (\%)} & \textit{n.s.} & \textit{n.s.} & \textit{n.s.} & \textit{n.s.} \\
Female & 1187 (66.2) & 84 (72.4) & 16 (30.8) & 39 (43.3) \\
Male   & 595  (33.2) & 32 (27.6) & 36 (69.2) & 48 (53.3) \\
Other  & 11   (0.6)  & 0  (0.0)  & 0  (0.0)  & 3  (3.3)  \\

\textbf{Education, n (\%)} & \textit{n.s.} & \textit{n.s.} & \textit{n.s.} & \textit{n.s.} \\
No diploma   & 52   (2.9)  & 11  (9.5)  & 7  (13.5)  & -  \\
Secondary    & 291  (16.2) & 37  (31.9) & 8  (15.4)  & - \\
Higher short & 213  (11.9) & 52  (44.8) & 0  (0.0)   & - \\
Higher long  & 1236 (69.0) & 16  (13.8) & 37 (71.2)  & - \\
\end{tabular}}
%\caption{Demographics and clinical scores of the four cohorts. Categorical variables are compared with the Pearson chi-square test, and continuous variables are compared with the Kruskal-Wallis $H$ test based on control and non-control groups (***: $p<0.001$, \textit{n.s.} means not significant). For example, }
\label{tab:demographics}
\end{table}

\end{frame}

\begin{frame}{Clinical evaluation}

\begin{table}[!htbp]
\centering
\resizebox{1\textwidth}{!}{
\begin{tabular}{p{4cm}llll}
 &
\textbf{\shortstack[l]{General\\Population}} &
\textbf{Androids} &
\textbf{MODMA} &
\textbf{VOCES} \\
 &
\textbf{n=1809} &
\textbf{n=116} &
\textbf{n=52} &
\textbf{n=90} \\

\hline
\textbf{C-SSRS} & \textit{n.s.}  & \textit{n.s.}  & \textit{n.s.}  & \textit{n.s.} \\
Suicidal risk, n (\%) & - & - & - & 60 (66.7) \\
No suicidal risk, n (\%) & - & - & - & 30 (33.3) \\
\textbf{MADRS / MDD} & \textit{n.s.} &  \textit{n.s.}  & \textit{n.s.}   & \textit{n.s.}   \\
Depression, n (\%) & -  & 64 (55.2) & 23 (44.2) & -  \\
No depression, n (\%) & -  & 52 (44.8) & 29 (55.8) & -  \\
%\textbf{AIS} & *** & & & \\
%Mean (SD) & 5.2 (3.9) & & & \\
%Range & 0--24 & & & \\
%
%\textbf{BDI} & *** & & & \\
%Mean (SD) & 7.4 (7.8) &  &  &  \\
%Range     & 0--54      &  &  &  \\

%\textbf{GAD-7} & *** & & *** & \\
%Mean (SD) & 4.8 (4.5) & & 7.3 (7.1) &  \\
%Range     & 0--21      & & 0--21     &  \\
%
%\textbf{MFI} & *** & & & \\
%Mean (SD) & 44.3 (15.9) &  &  &  \\
%Range     & 18--99       &  &  &  \\

\textbf{PHQ-9} & \textit{n.s.}  & \textit{n.s.}  & *** & *** \\
Mean (SD) & 5.2 (4.6) & -  & 9.4 (8.5) & 10.5 (6.8) \\
Range     & 0--27      & -  & 0--25     & 0.0--26.0 \\
\end{tabular}}
%\caption{Demographics and clinical scores of the four cohorts. Categorical variables are compared with the Pearson chi-square test, and continuous variables are compared with the Kruskal-Wallis $H$ test based on control and non-control groups (***: $p<0.001$, \textit{n.s.} means not significant). For example, }
\label{tab:demographics}
\end{table}

\end{frame}

\begin{comment}

\begin{frame}{Oneirogen, a language model for dream generation}  

Oneirogen (\href{https://huggingface.co/gustavecortal/oneirogen-0.5B}{0.5}, \href{https://huggingface.co/gustavecortal/oneirogen-1.5B}{1.5}, \href{https://huggingface.co/gustavecortal/oneirogen-7B}{7B}), a language model for dream generation. It is based on \href{https://huggingface.co/Qwen/Qwen2-7B}{Qwen2} and was trained on \href{https://dreambank.net/}{DreamBank}

\vspace{0.25cm}
\pause

\noindent Oneirogen was used to produce \href{https://huggingface.co/datasets/gustavecortal/the-android-and-the-human}{The Android and The Machine}, an English dataset composed of 10,000 real and 10,000 generated dreams


\vspace{0.25cm}
%\pause

\textit{I'm in a building that seems to be a school or maybe a university. There is a lot of noise and activity, and everyone is very busy talking. It is very loud and unpleasant - too loud to talk to anyone easily. The walls are made out of some soft material that might be plastic foam.}

\vspace{0.25cm}

\textit{I was at a shop. There were lots of people there and I lost Mom and Ezra. Later, we were in a car park. We went to get pizza's for dinner from the nearby pizza place but it was really late so they wouldn't serve us. [I think I was also walking around the shops earlier].}

\end{frame}

\begin{frame}{Post-training for psychology}

Piaget (\href{https://huggingface.co/gustavecortal/Piaget-0.6B}{0.6}, \href{https://huggingface.co/gustavecortal/Piaget-1.7B}{1.7}, \href{https://huggingface.co/gustavecortal/Piaget-4B}{4}, \href{https://huggingface.co/gustavecortal/Piaget-8B}{8B}), a model for psychological reasoning

\smallskip
\pause

Domain filtering on open reasoning traces from \href{https://huggingface.co/datasets/cognitivecomputations/dolphin-r1}{Dolphin R1} and \href{https://huggingface.co/datasets/GeneralReasoning/GeneralThought-430K}{General Reasoning}

\smallskip
\pause

Prompts were embedded, clustered with $k$-means ($k=20\,000$) and majority-voted for domain labels using \href{https://huggingface.co/Qwen/Qwen3-1.7B}{Qwen3-1.7B}%, following the \href{https://huggingface.co/Intelligent-Internet/II-Medical-8B-1706}{Intelligent Internet pipeline}. 

\smallskip
\pause

Clusters tagged psychology or philosophy were retained for LoRA finetuning based on Qwen3% ($\text{rank}=8$, $\alpha=16$, $\text{max length}=2048$, $\text{epoch}=1$, $\text{batch size}=16$).

\smallskip

\begin{figure}
    \centering
    \includegraphics[scale=0.15]{img/lora.png}
    %\caption{Examples of reasoning traces used for post-training Piaget.}
    \label{fig:psychological_reasoning_examples}
\end{figure}

\end{frame}

\begin{frame}{Post-training for psychology}


Beck (\href{https://huggingface.co/gustavecortal/Beck-0.6B}{0.6}, \href{https://huggingface.co/gustavecortal/Beck-1.7B}{1.7}, \href{https://huggingface.co/gustavecortal/Beck-4B}{4}, \href{https://huggingface.co/gustavecortal/Beck-8B}{8B}), a model aligned with psychotherapeutic preferences

\smallskip
\pause

Beck is based on \href{https://huggingface.co/gustavecortal/Piaget-8B}{Piaget} and was finetuned on psychotherapeutic preferences from \href{https://huggingface.co/datasets/Psychotherapy-LLM/PsychoCounsel-Preference}{PsychoCounsel-Preference} using preference optimization (ORPO) and LoRA

\smallskip
\pause

\begin{figure}
    \centering
    \includegraphics[scale=0.17]{img/orpo.png}
    %\caption{Examples of reasoning traces used for post-training Piaget.}
    \label{fig:orpo}
\end{figure}


\end{frame}

\begin{frame}{How does it compare to other language models?}
    % The Prompt
    \begin{block}{User prompt}
        \scriptsize
        Anxiety often strikes when I'm faced with making decisions. 
        The fear of making the wrong choice or disappointing others paralyzes me, leaving me stuck in indecision. 
        I want to learn how to trust myself and make confident choices.% \texttt{\textbackslash no\_think}
    \end{block}

    \vspace{0.2cm}

    % Comparison Columns
    \begin{columns}[T] % T aligns columns at the top
        
        % Qwen3-8B Column
        \begin{column}{0.48\textwidth}
            \textbf{\footnotesize Qwen3-8B} \\
            \scriptsize
            It's completely natural to feel anxious when facing important decisions, especially when you're worried about making the wrong choice or letting others down. This internal conflict can create a sense of overwhelm that makes it hard to move forward. Learning to trust yourself is a powerful step toward overcoming this challenge. Start by reflecting on past decisions—what worked, what didn't, and what you learned from them. Over time, you'll begin to recognize your own values, strengths, and patterns of thinking. Practicing self-compassion is also key; remind yourself that no decision is perfect and that growth comes from experience, not perfection [...]% Small steps, like setting aside time each day for reflection or journaling, can help build confidence. Surround yourself with people who support your journey, and don't hesitate to seek guidance if needed. Trusting yourself isn’t about having all the answers—it’s about believing in your ability to navigate uncertainty and grow through the process. With patience and practice, you’ll find greater clarity and courage in your choices.
        \end{column}

        % Beck-8B Column
        \begin{column}{0.48\textwidth}
            \textbf{\footnotesize Beck-8B} \\
            \scriptsize
            It sounds like you're dealing with a lot of pressure regarding decision-making and it's causing you significant distress. This is quite common for many people, as we all have moments where choosing can feel overwhelming. Let’s explore what might be contributing to these feelings so that we can work on building your confidence in decision-making.

            Firstly, consider what specific fears arise when you think about making a decision. Is there a particular aspect of the choice that triggers anxiety? Sometimes identifying just one component can help in breaking down the process.

            Also, think about past decisions that went well, even if they don’t seem big at first [...]% Remembering successful outcomes can reinforce trust in yourself, reminding you that you have experience navigating through uncertainties successfully before.%

            %Building self-trust involves challenging any negative thoughts surrounding decision-making. It may also help to practice mindfulness—being fully present during each step of the decision process rather than anticipating worst-case scenarios. By cultivating patience and acceptance towards ambiguity, you might find that some anxieties lessen over time.

            %Would you like us to brainstorm practical exercises or strategies tailored specifically for boosting your decision-making abilities and confidence?

        \end{column}
        
    \end{columns}
\end{frame}

\end{comment}

\section{References}
\begin{frame}{}
\Large
\vfill
\begin{center}
    References
\end{center}
\vfill

\end{frame}

\begin{frame}[allowframebreaks]{References}
\footnotesize
\printbibliography
\end{frame}

\end{document}
