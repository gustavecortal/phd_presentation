% --begin pap's style
\documentclass[10pt]{beamer}

%\usetheme[secheader]{Darmstadt}
%\usetheme{Pittsburgh}

\usepackage[english]{babel}


\usepackage[utf8]{inputenc}
\usepackage[final]{pdfpages}
\usepackage[official]{eurosym}
\usepackage{graphicx}
\usepackage{caption}
\usepackage{csquotes}
\usepackage{multirow}
\usepackage{amsmath}
\usepackage{amsfonts} % for \text
\usepackage{hyperref}
\usepackage{comment}
\usepackage{graphicx}
\usepackage{booktabs}
\usepackage{tabularx}
\usepackage[flushleft]{threeparttable}  % for table notes
\newcommand\mscriptsize[1]{\mbox{\scriptsize\ensuremath{#1}}}
\newcommand\mtiny[1]{\mbox{\tiny\ensuremath{#1}}}
\usepackage{xurl}             % allow line breaks in long URLs
\usepackage{ragged2e}         % better ragged-right with hyphenation
\usepackage{xspace}           % for intelligent spacing after commands
\setlength{\emergencystretch}{3em} % gentle last-resort stretch

\usepackage{subcaption}

\usepackage[T1]{fontenc}
\usepackage[utf8]{inputenc}
%\usepackage{minted}           % core package
\usepackage{xcolor}           % for background color
\definecolor{LightGray}{gray}{0.95}


\definecolor{blueNCS}{rgb}{0.0, 0.53, 0.74}    % ✓ Commas
\definecolor{bluepanam}{rgb}{0.0, 0.189, 0.79} % ✓ Commas
\definecolor{Aplgreen}{rgb}{0.55, 0.71, 0.0}   % ✓ Commas
\definecolor{DarkBlue}{rgb}{0.1,0.1,0.9}

\hypersetup{
    colorlinks=false,
    linkcolor=bluepanam,
    filecolor=bluepanam,      
    urlcolor=bluepanam,
    pdftitle={Overleaf Example},
    pdfpagemode=FullScreen,
    }

\setbeamercolor{palette primary}{bg=white,fg=black}
\setbeamercolor{palette secondary}{bg=white,fg=bluepanam}
\setbeamercolor{palette tertiary}{bg=white,fg=bluepanam}
\setbeamercolor{palette quaternary}{bg=white,fg=bluepanam}
\setbeamercolor{structure}{fg=bluepanam} % itemize, enumerate, etc
\setbeamercolor{section in toc}{fg=bluepanam} % TOC sections


%% \setbeamercolor{section in head/foot}{fg=white, bg=blue}
\setbeamercolor{title}{fg=bluepanam, bg=white}
\setbeamercolor{author}{fg=black, bg=white}
\setbeamercolor{institute}{fg=black, bg=white}
\setbeamercolor{date}{fg=white, bg=white}


\setbeamercolor{section in head/foot}{fg=bluepanam, bg=white}     
     
\setbeamercolor{author in head/foot}{fg=bluepanam, bg=white}
%\setbeamercolor{author in head/foot}{fg=white, bg=white}
\beamertemplatenavigationsymbolsempty
% --end pap's style

%\newcommand{\customcite}[1]{\citeauthor{#1}, \citetitle{#1}, \citeyear{#1}, \citeurl{#1}}
%\newcommand{\customcitenourl}[1]{\citeauthor{#1}, \citetitle{#1}, \citeyear{#1}}
\newcommand{\parensmall}[1]{{\scriptsize #1}}

% Custom fullcite command without URL, DOI, and urldate
\newcommand{\fullcitenourl}[1]{%
  \AtNextCite{%
    \clearfield{url}%
    \clearfield{doi}%
    \clearfield{urldate}%
  }%
  \fullcite{#1}%
}


\usepackage{xurl}             % allow line breaks in long URLs
\usepackage{ragged2e}         % better ragged-right with hyphenation
\usepackage{xspace}           % for intelligent spacing after commands
\setlength{\emergencystretch}{3em} % gentle last-resort stretch

%\newcommand{\comment}[1]{}

\usepackage[style=authoryear,natbib=true,backend=biber,date=year,urldate=long]{biblatex}
\addbibresource{phd_slides_biblatex.bib}

% Remove "visited on" dates from URLs
\AtEveryBibitem{\clearfield{urldate}}
\hypersetup{
  colorlinks=true,
  linkcolor=bluepanam,
  filecolor=bluepanam,      
  urlcolor=bluepanam,
  citecolor=black,  % Changed to black to match text color
  pdftitle={Overleaf Example},
  pdfpagemode=FullScreen,
}

% Add more spacing between entries
\setlength\bibitemsep{0.5\baselineskip}

\title{Automatic thematic analysis in mental health narratives using language models}
\author{Gustave Cortal}
\institute{\footnotesize Deezer technical interview on a Github repository}
%\titlegraphic{
%  \includegraphics[width=4cm]{img/lmf_logo_emnlp.png}
%  \hspace{1cm}
%  \includegraphics[width=5cm]{img/logo_ens_saclay.png}
%}
\date{\today}

\makeatletter
\defbeamertemplate*{footline}{myminiframes theme}
  {%
    \begin{beamercolorbox}[colsep=1.5pt]{upper separation line foot}
    \end{beamercolorbox}
    \begin{beamercolorbox}[ht=2.5ex,dp=1.125ex,%
      leftskip=.3cm,rightskip=.3cm plus1fil]{author in head/foot}%
      \leavevmode{\usebeamerfont{author in head/foot}}%
      \hfill%
    %\insertframenumber{}\,/\,\inserttotalframenumber%
    \end{beamercolorbox}%
    \begin{beamercolorbox}[ht=2.5ex,dp=2.125ex,leftskip=.3cm,rightskip=.3cm plus1fil]{section in head/foot}%
      %%      {\usebeamerfont{section in head/foot} somthing written here \hfill  \setlength{\fboxrule}{0pt}\setlength{\fboxsep}{0pt}\fcolorbox{blueNCS}{blueNCS!70}{My own image here}}%
      {\usebeamerfont{section in head/foot} \insertshortauthor \hfill  %\setlength{\fboxrule}{0pt}\setlength{\fboxsep}{0pt}\fcolorbox{blueNCS}{blueNCS!100}{foobar etc}
      }%
      \insertframenumber{}\,/\,\inserttotalframenumber%
    \end{beamercolorbox}%
    \begin{beamercolorbox}[colsep=1.5pt]{lower separation line foot}
    \end{beamercolorbox}
  }
  \makeatother


\begin{document}

\setlength{\parskip}{5pt}%
\setlength{\parsep}{0pt}%
\setlength{\itemsep}{0.25cm}%
\setlength{\leftmargini}{0.5cm}

\begin{frame}
  \titlepage
\end{frame}

\begin{comment}

  \begin{frame}{}
\Large
\begin{center}
    Automatic thematic analysis in mental health narratives using language models% (change title to thematic analysis?)
    \section{Thematic analysis in mental health narratives using language models}
\end{center}

\vspace{1.5cm}

\footnotesize

\textbf{G. Cortal}, S. Guessoum, X. Cao, R. Riad. \textit{Fine-grained mental health topic modeling in different cohorts using large language models} (preprint). 2025. 

\end{frame}
  
\end{comment}

\begin{frame}{Introduction}

  \begin{itemize}[<+->]
    \item Qualitative analysis of speech content is central to clinical practice
    \item Thematic analysis studies how people construct meaning% from their experiences
    \item Thematic analysis is time-consuming, and typically constrained to small, monolingual corpora
    \item Computational approaches offers time savings over manual annotation, and the power to analyze a larger amount of data
  \end{itemize}

  \pause
  \vspace{0.5cm}

  $\rightarrow$ We developed a pipeline that (a) clusters narratives from different cohorts, (b) generates descriptions for each cluster, and (c) links clusters to variation in clinical scores and sociodemographic factors %We then quantified which questions best separate clinically relevant clusters, and we conducted an analysis of confounding by age, education, and sex, an aspect often overlooked in topic modeling despite strong demographic effects on language use.

\end{frame}

\begin{frame}{Data collection}

We collected clinical scores and narratives from \textbf{four cohorts}. A French general population cohort (n=1809), and three clinical populations: Italian (n=116), Chinese (n=52), and Spanish (n=90) cohorts%. We focus on the French general population cohort.

\vspace{0.5cm}
\pause

\textbf{Clinical scores} were assessed using various scales such as: AIS (Athens Insomnia Scale); BDI (Beck Depression Inventory); GAD-7 (Generalized Anxiety Disorder 7-item scale); MFI (Multidimensional Fatigue Inventory); PHQ-9 (Patient Health Questionnaire-9 for depression)

\vspace{0.5cm}
\pause

\textbf{Open-ended questions}: \textit{Describe your last 24 hours} / \textit{a negative event that happened to you in the past} / \textit{a positive event that happened to you in the past} / \textit{a negative event you think might happen in the future} / \textit{a positive event you think might happen in the future} / \textit{Describe how you are feeling at the moment and how your sleep has been lately}

\end{frame}

\begin{frame}{Pipeline for semantic clustering and description generation}

  %We developed a multilingual pipeline that (a) clusters spontaneous speech transcripts from four cohorts (general population and three clinical samples in French, Italian, Chinese, and Spanish), (b) generates fine-grained natural-language descriptions for each cluster, and (c) links clusters to variation in clinical scores and sociodemographic factors. We then quantified which questions best separate clinically relevant clusters, and we conducted an analysis of confounding by age, education, and sex, an aspect often overlooked in topic modeling despite strong demographic effects on language use. 

\begin{figure}
    \centering
    \includegraphics[scale=0.6]{img/topic_modeling/methods_v3.png}
    %\caption{\textbf{Computational pipeline for semantic clustering and description generation}. Speech transcripts are converted to semantic vectors via multilingual embeddings, dimensionally reduced, and grouped into clusters using density-based methods. Each cluster undergoes: (1) statistical analysis linking cluster membership to clinical/demographic  variables, and (2) automatic description generation where a large language model summarizes transcripts per cluster into human-readable natural language. This dual quantitative-qualitative output enables both statistical hypothesis testing and clinical interpretation of discovered topics.}
    %\caption{Each cluster undergoes: (1) statistical analysis linking cluster membership to clinical/demographic  variables, and (2) automatic description generation where a large language model summarizes transcripts per cluster into human-readable natural language.}
    \label{fig:methods}
\end{figure}

\end{frame}

\begin{frame}{Distribution of depression scores across clusters}

  \textit{How you are feeling and how your sleep has been lately}% (n=1,786 transcripts) 
  \begin{figure}
    \centering
    \includegraphics[scale=0.06]{img/topic_modeling/boxplot_description/popgen_description_larger_boxplots.png}
    %\caption{Distribution of PHQ-9 depression scores across automatically identified clusters in the French general population cohort (n=1,786 transcripts). Responses to \textit{Describe how you are feeling at the moment and how your sleep has been lately}.}
    %\caption{Distribution of PHQ-9 depression scores across automatically identified clusters in the French general population cohort (n=1,786 transcripts). Responses to the prompt 'Describe how you are feeling at the moment and how your sleep has been lately'. Box plots show median (center line), interquartile range (box), and 1.5x IQR whiskers. Effect size (Kruskal-Wallis H = 0.17, p < 0.00001) indicates moderate discrimination between clusters. Sample sizes per cluster range from n=34 to n=92. Selected cluster descriptions (Clusters 1, 10, 12, 26) were generated by Qwen3-14B summarizing 30 random transcripts per cluster.}
    \label{fig:popgen_description}
\end{figure}


%\vspace{0.1cm}
\pause

$\rightarrow$ Depression scores vary significantly: cluster 26 highest (13.4±5.4), cluster 1 lowest (2.6±2.2)


%For the French general population cohort based on answers to “Describe how you’re feeling and how your nights have been”, among 26 identified clusters (n=1,786 transcripts), PHQ-9 scores differed significantly across clusters $(H=0.17, p\leq0.00001)$. Cluster 26 (n=37, age=25±9) exhibited the highest depression scores PHQ-9 (13.4±5.4) with LLM-generated descriptions highlighting \textit{sleep disturbances characterized by insomnia, frequent awakenings, and restless sleep}, alongside pervasive \textit{anxiety, emotional instability, and self-esteem issues}. In contrast, Cluster 1 (n=35, age=39±19) showed the lowest scores PHQ-9 (2.6±2.2), with descriptions emphasizing\textit{ consistent satisfaction with current well-being, good sleep quality, and general relaxation}.

%Distribution of PHQ-9 depression scores across automatically identified clusters in the French general population cohort
\end{frame}

\begin{frame}{Generated cluster descriptions}
  \begin{figure}
    \centering
    \includegraphics[scale=0.25]{img/topic_modeling/boxplot_description/popgen_description_larger_descriptions.png}
    %\caption{Distribution of PHQ-9 depression scores across automatically identified clusters in the French general population cohort (n=1,786 transcripts). Responses to the prompt 'Describe how you are feeling at the moment and how your sleep has been lately'. Box plots show median (center line), interquartile range (box), and 1.5x IQR whiskers. Effect size (Kruskal-Wallis H = 0.17, p < 0.00001) indicates moderate discrimination between clusters. Sample sizes per cluster range from n=34 to n=92. Selected cluster descriptions (Clusters 1, 10, 12, 26) were generated by Qwen3-14B summarizing 30 random transcripts per cluster.}
    \label{fig:popgen_description}
\end{figure}

\pause

$\rightarrow$ Clustering captures symptom severity and age-related circumstances

%Selected cluster descriptions (Clusters 1, 10, 12, 26) were generated by Qwen3-14B summarizing 30 random transcripts per cluster.
\end{frame}

\begin{frame}{Effect size across questions and clinical scores}

\begin{figure}
    \centering
    \includegraphics[scale=0.35]{img/topic_modeling/heatmap_effect_sizes/V5_V6_V7_V8_V9_V10_phq9_gad7_bdi_ais_mfi_global_heatmap.png}
    %\caption{Effect size across questions and clinical scores for the French general population cohort}
    \label{fig:popgen_clinical_heatmap}
\end{figure}

\pause
$\rightarrow$ Certain questions better discriminate clinical scores

\end{frame}

\begin{frame}{Effect size across questions and sociodemographics}

\begin{figure}
  \centering
  \includegraphics[scale=0.35]{img/topic_modeling/heatmap_effect_sizes/V5_V6_V7_V8_V9_V10_diploma_level_gender_age_global_heatmap.png}
  %\caption{Effect size across questions and sociodemographics for the French general population cohort}
  \label{fig:popgen_demo_heatmap}
\end{figure}

\pause

$\rightarrow$ Nearly all questions discriminate sociodemographics

\end{frame}

\begin{frame}{}
\Large
\begin{center}
    Appendix
    \section{Appendix}
\end{center}

\end{frame}

\begin{frame}{Demographics}

\begin{table}[!htbp]
\centering
\resizebox{1\textwidth}{!}{
\begin{tabular}{p{4cm}llll}
 &
\textbf{\shortstack[l]{General\\Population}} &
\textbf{Androids} &
\textbf{MODMA} &
\textbf{VOCES} \\
 &
\textbf{n=1809} &
\textbf{n=116} &
\textbf{n=52} &
\textbf{n=90} \\

\hline
\textbf{Demographics} & & & & \\

\textbf{Language} & French & Italian & Chinese & Spanish \\

\textbf{Age} & *** & \textit{n.s.} & \textit{n.s.} & *** \\
Mean (SD) & 37.8 (18.2) & 37.4 (12.0) & 31.3 (9.2) & 38.6 (14.9) \\
Range     & 18--91       & 19--71      & 18--52      & 21--76       \\

\textbf{Sex, n (\%)} & \textit{n.s.} & \textit{n.s.} & \textit{n.s.} & \textit{n.s.} \\
Female & 1187 (66.2) & 84 (72.4) & 16 (30.8) & 39 (43.3) \\
Male   & 595  (33.2) & 32 (27.6) & 36 (69.2) & 48 (53.3) \\
Other  & 11   (0.6)  & 0  (0.0)  & 0  (0.0)  & 3  (3.3)  \\

\textbf{Education, n (\%)} & \textit{n.s.} & \textit{n.s.} & \textit{n.s.} & \textit{n.s.} \\
No diploma   & 52   (2.9)  & 11  (9.5)  & 7  (13.5)  & -  \\
Secondary    & 291  (16.2) & 37  (31.9) & 8  (15.4)  & - \\
Higher short & 213  (11.9) & 52  (44.8) & 0  (0.0)   & - \\
Higher long  & 1236 (69.0) & 16  (13.8) & 37 (71.2)  & - \\
\end{tabular}}
%\caption{Demographics and clinical scores of the four cohorts. Categorical variables are compared with the Pearson chi-square test, and continuous variables are compared with the Kruskal-Wallis $H$ test based on control and non-control groups (***: $p<0.001$, \textit{n.s.} means not significant). For example, }
\label{tab:demographics}
\end{table}

\end{frame}

\begin{frame}{Clinical evaluation}

\begin{table}[!htbp]
\centering
\resizebox{1\textwidth}{!}{
\begin{tabular}{p{4cm}llll}
 &
\textbf{\shortstack[l]{General\\Population}} &
\textbf{Androids} &
\textbf{MODMA} &
\textbf{VOCES} \\
 &
\textbf{n=1809} &
\textbf{n=116} &
\textbf{n=52} &
\textbf{n=90} \\

\hline
\textbf{C-SSRS} & \textit{n.s.}  & \textit{n.s.}  & \textit{n.s.}  & \textit{n.s.} \\
Suicidal risk, n (\%) & - & - & - & 60 (66.7) \\
No suicidal risk, n (\%) & - & - & - & 30 (33.3) \\
\textbf{MADRS / MDD} & \textit{n.s.} &  \textit{n.s.}  & \textit{n.s.}   & \textit{n.s.}   \\
Depression, n (\%) & -  & 64 (55.2) & 23 (44.2) & -  \\
No depression, n (\%) & -  & 52 (44.8) & 29 (55.8) & -  \\
%\textbf{AIS} & *** & & & \\
%Mean (SD) & 5.2 (3.9) & & & \\
%Range & 0--24 & & & \\
%
%\textbf{BDI} & *** & & & \\
%Mean (SD) & 7.4 (7.8) &  &  &  \\
%Range     & 0--54      &  &  &  \\

%\textbf{GAD-7} & *** & & *** & \\
%Mean (SD) & 4.8 (4.5) & & 7.3 (7.1) &  \\
%Range     & 0--21      & & 0--21     &  \\
%
%\textbf{MFI} & *** & & & \\
%Mean (SD) & 44.3 (15.9) &  &  &  \\
%Range     & 18--99       &  &  &  \\

\textbf{PHQ-9} & \textit{n.s.}  & \textit{n.s.}  & *** & *** \\
Mean (SD) & 5.2 (4.6) & -  & 9.4 (8.5) & 10.5 (6.8) \\
Range     & 0--27      & -  & 0--25     & 0.0--26.0 \\
\end{tabular}}
%\caption{Demographics and clinical scores of the four cohorts. Categorical variables are compared with the Pearson chi-square test, and continuous variables are compared with the Kruskal-Wallis $H$ test based on control and non-control groups (***: $p<0.001$, \textit{n.s.} means not significant). For example, }
\label{tab:demographics}
\end{table}

\end{frame}

\end{document}
