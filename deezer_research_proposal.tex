% --begin pap's style
\documentclass[10pt]{beamer}

%\usetheme[secheader]{Darmstadt}
%\usetheme{Pittsburgh}

\usepackage[english]{babel}


\usepackage[utf8]{inputenc}
\usepackage[final]{pdfpages}
\usepackage[official]{eurosym}
\usepackage{graphicx}
\usepackage{caption}
\usepackage{csquotes}
\usepackage{multirow}
\usepackage{amsmath}
\usepackage{amsfonts} % for \text
\usepackage{hyperref}
\usepackage{comment}
\usepackage{graphicx}
\usepackage{booktabs}
\usepackage{tabularx}
\usepackage[flushleft]{threeparttable}  % for table notes
\newcommand\mscriptsize[1]{\mbox{\scriptsize\ensuremath{#1}}}
\newcommand\mtiny[1]{\mbox{\tiny\ensuremath{#1}}}
\usepackage{xurl}             % allow line breaks in long URLs
\usepackage{ragged2e}         % better ragged-right with hyphenation
\usepackage{xspace}           % for intelligent spacing after commands
\setlength{\emergencystretch}{3em} % gentle last-resort stretch

\usepackage{subcaption}

\usepackage[T1]{fontenc}
\usepackage[utf8]{inputenc}
%\usepackage{minted}           % core package
\usepackage{xcolor}           % for background color
\definecolor{LightGray}{gray}{0.95}


\definecolor{blueNCS}{rgb}{0.0, 0.53, 0.74}    % ✓ Commas
\definecolor{bluepanam}{rgb}{0.0, 0.189, 0.79} % ✓ Commas
\definecolor{Aplgreen}{rgb}{0.55, 0.71, 0.0}   % ✓ Commas
\definecolor{DarkBlue}{rgb}{0.1,0.1,0.9}

\hypersetup{
    colorlinks=false,
    linkcolor=bluepanam,
    filecolor=bluepanam,      
    urlcolor=bluepanam,
    pdftitle={Overleaf Example},
    pdfpagemode=FullScreen,
    }

\setbeamercolor{palette primary}{bg=white,fg=black}
\setbeamercolor{palette secondary}{bg=white,fg=bluepanam}
\setbeamercolor{palette tertiary}{bg=white,fg=bluepanam}
\setbeamercolor{palette quaternary}{bg=white,fg=bluepanam}
\setbeamercolor{structure}{fg=bluepanam} % itemize, enumerate, etc
\setbeamercolor{section in toc}{fg=bluepanam} % TOC sections


%% \setbeamercolor{section in head/foot}{fg=white, bg=blue}
\setbeamercolor{title}{fg=bluepanam, bg=white}
\setbeamercolor{author}{fg=black, bg=white}
\setbeamercolor{institute}{fg=black, bg=white}
\setbeamercolor{date}{fg=white, bg=white}


\setbeamercolor{section in head/foot}{fg=bluepanam, bg=white}     
     
\setbeamercolor{author in head/foot}{fg=bluepanam, bg=white}
%\setbeamercolor{author in head/foot}{fg=white, bg=white}
\beamertemplatenavigationsymbolsempty
% --end pap's style

%\newcommand{\customcite}[1]{\citeauthor{#1}, \citetitle{#1}, \citeyear{#1}, \citeurl{#1}}
%\newcommand{\customcitenourl}[1]{\citeauthor{#1}, \citetitle{#1}, \citeyear{#1}}
\newcommand{\parensmall}[1]{{\scriptsize #1}}

% Custom fullcite command without URL, DOI, and urldate
\newcommand{\fullcitenourl}[1]{%
  \AtNextCite{%
    \clearfield{url}%
    \clearfield{doi}%
    \clearfield{urldate}%
  }%
  \fullcite{#1}%
}


\usepackage{xurl}             % allow line breaks in long URLs
\usepackage{ragged2e}         % better ragged-right with hyphenation
\usepackage{xspace}           % for intelligent spacing after commands
\setlength{\emergencystretch}{3em} % gentle last-resort stretch

%\newcommand{\comment}[1]{}

\usepackage[style=authoryear,natbib=true,backend=biber,date=year,urldate=long]{biblatex}
\addbibresource{phd_slides_biblatex.bib}

% Remove "visited on" dates from URLs
\AtEveryBibitem{\clearfield{urldate}}
\hypersetup{
  colorlinks=true,
  linkcolor=bluepanam,
  filecolor=bluepanam,      
  urlcolor=bluepanam,
  citecolor=black,  % Changed to black to match text color
  pdftitle={Overleaf Example},
  pdfpagemode=FullScreen,
}

% Add more spacing between entries
\setlength\bibitemsep{0.5\baselineskip}

\title{Language-based representations\\ for music recommendation}
\author{Gustave Cortal}
\institute{\footnotesize Deezer research proposal}
%\titlegraphic{
%  \includegraphics[width=4cm]{img/lmf_logo_emnlp.png}
%  \hspace{1cm}
%  \includegraphics[width=5cm]{img/logo_ens_saclay.png}
%}
\date{\today}

\makeatletter
\defbeamertemplate*{footline}{myminiframes theme}
  {%
    \begin{beamercolorbox}[colsep=1.5pt]{upper separation line foot}
    \end{beamercolorbox}
    \begin{beamercolorbox}[ht=2.5ex,dp=1.125ex,%
      leftskip=.3cm,rightskip=.3cm plus1fil]{author in head/foot}%
      \leavevmode{\usebeamerfont{author in head/foot}}%
      \hfill%
    %\insertframenumber{}\,/\,\inserttotalframenumber%
    \end{beamercolorbox}%
    \begin{beamercolorbox}[ht=2.5ex,dp=2.125ex,leftskip=.3cm,rightskip=.3cm plus1fil]{section in head/foot}%
      %%      {\usebeamerfont{section in head/foot} somthing written here \hfill  \setlength{\fboxrule}{0pt}\setlength{\fboxsep}{0pt}\fcolorbox{blueNCS}{blueNCS!70}{My own image here}}%
      {\usebeamerfont{section in head/foot} \insertshortauthor \hfill  %\setlength{\fboxrule}{0pt}\setlength{\fboxsep}{0pt}\fcolorbox{blueNCS}{blueNCS!100}{foobar etc}
      }%
      \insertframenumber{}\,/\,\inserttotalframenumber%
    \end{beamercolorbox}%
    \begin{beamercolorbox}[colsep=1.5pt]{lower separation line foot}
    \end{beamercolorbox}
  }
  \makeatother


\begin{document}

\setlength{\parskip}{5pt}%
\setlength{\parsep}{0pt}%
\setlength{\itemsep}{0.25cm}%
\setlength{\leftmargini}{0.5cm}

\begin{frame}
  \titlepage
\end{frame}

\begin{comment}

  \begin{frame}{}
\Large
\begin{center}
    Automatic thematic analysis in mental health narratives using language models% (change title to thematic analysis?)
    \section{Thematic analysis in mental health narratives using language models}
\end{center}

\vspace{1.5cm}

\footnotesize

\textbf{G. Cortal}, S. Guessoum, X. Cao, R. Riad. \textit{Fine-grained mental health topic modeling in different cohorts using large language models} (preprint). 2025. 

\end{frame}
  
\end{comment}

\begin{frame}{Context and objective}

    \begin{itemize}[<+->]
    \item The research proposal aims to develop language-based representations for musical items and users
    \item Large Language Models (LLMs) can be used to generate textual descriptions of items and user profiles
    \item This enable explainable and flexible music recommendation systems
    \item The research strategy is articulated around three interconnected axes: user modeling, item modeling, and music recommendation
  \end{itemize}

  %\vspace{0.5cm}
  %\pause

  %\textit{I am still on the exploration phase, I want your feedback on the directions}

  % Some research directions are too general, too specific, or not feasible with current LLMs and data

\end{frame}

\begin{frame}{}
\Large
\begin{center}
    User modeling
    \section{User modeling}
\end{center}

\end{frame}

\begin{frame}{User modeling}

\textit{How can we leverage LLMs to obtain scrutable natural language profiles?}

    \vspace{0.25cm}
  \pause

      \begin{itemize}[<+->]
    \item Preference modeling has relied on interaction data to produce embeddings
    \item Vector-based methods lack interpretability and raise scrutability issues
    \item Scientific literature establishes a strong link between musical preferences and psychological traits
    \item We could use LLMs to infer these states and preferences, transforming them into natural language profiles
    \item Natural language descriptions serve as a transparent base that the user can verify and refine, and provide a context for the recommendation engine
  \end{itemize}

    %\vspace{0.5cm}
  %\pause

\end{frame}

\begin{frame}{User modeling}

\begin{figure}
    \centering
    \includegraphics[scale=0.23]{deezer_img/nl_user_preference_profiles.png}
    %\caption{User modeling diagram}
    \label{fig:user_modeling_diagram}
\end{figure}

    \vspace{0.25cm}
  \pause

  $\rightarrow$ Natural language profiles involve preferences, listening intents, personality, mood and emotion

% long-term preferences vs short-term intents

\end{frame}

\begin{frame}{User modeling}

  \textit{How can we better modelize user preferences, especially regarding their affective states?}

      \vspace{0.25cm}
  \pause


  \begin{itemize}[<+->]
    \item Existing research has explored mood and emotion detection from music listening data
    \item However, current models often rely on simplistic emotion categories (e.g., discrete emotions or valence-arousal dimensions)
    \item There is a need for more nuanced representations that capture the complexity of human emotions
    \item Incorporating cognitive science theories of emotion can enhance the understanding of user affective states
    %\item This can lead to more personalized and context-aware music recommendations
  \end{itemize}

        \vspace{0.25cm}
  \pause

  \textit{We could reuse some ideas to better modelize items' affective dimensions}

  %+ add emotion annotation schemes, cognitive science theories of emotion

  %A promising direction involves exploiting language models to generate synthetic training data by simulating realistic conversations about music~\cite{dohTALKPLAYMultimodalMusic2025}

  %This approach enables the creation of synthetic datasets for tasks such as intent detection, allowing for the training of robust systems tailored to specific personas.



\end{frame}

\begin{frame}{}
\Large
\begin{center}
    Linguistic and cognitive science theories
    %\section{Definition of objectives using cognitive science}
\end{center}

\end{frame}

\begin{frame}{Psychological theories and emotion annotation schemes}

  \pause

What are current limitations and interesting research directions?

%\pause
%\vspace{0.25cm}

%We review psychological theories of emotion and emotion annotation schemes in NLP

\pause
\vspace{0.25cm}

%What are current limitations? 

%[add refs to each theory and annotation schemes]

\begin{center}
\begin{tabular}{p{6.em}|p{10.em}|p{12.em}}
Psychological theories & In text, emotion is... & Example \\
\hline
Basic emotions theory & a \textbf{category} & "I love philosophy." $\rightarrow$ \texttt{joy} \\
Constructivist theories & a continuous value with an \textbf{affective} meaning & "His voice soothes me." $\rightarrow$ \texttt{valence} (4/5), \texttt{arousal} (1/5) \\
Appraisal theory & a continuous value with a \textbf{cognitive} meaning & "I received a surprise gift." $\rightarrow$ \texttt{sudden} (4/5), \texttt{control} (0/5) \\ 
& composed of \textbf{semantic roles} & "Louise (\texttt{experiencer}) was angry (\texttt{cue}) towards Paul (\texttt{target}), because he didn’t inform her (\texttt{cause})."
\end{tabular}
\end{center}


%\vspace{0.5cm}

%\scriptsize

%\textbf{G. Cortal} and C. Bonard. \href{https://aclanthology.org/2024.cmcl-1.23/}{Improving Language Models for Emotion Analysis: Insights from Cognitive Science}. \textit{CMCL, ACL 2024}.
    
\end{frame}

\begin{frame}{Which verbal signs are used to infer expressed emotions?}

%Which verbal signs are used to infer expressed emotions?

\pause
%\vspace{0.5cm}

Raphaël Micheli categorizes a range of linguistic markers into three \textit{emotion expression modes} \citep{micheliEsquisseDuneTypologie2013}. The emotion can be: 

%\cite{Micheli2014}
\pause
\vspace{0.5cm}

\begin{itemize}[<+->]
    \item \textit{labeled} explicitly with an emotional term ("I am \underline{sad}")
    \item \textit{shown} with utterance features such as interjections and punctuations ("\underline{Ah!} That's great\underline{!}")
    \item \textit{suggested} with the description of a situation which generally, in a given sociocultural context, leads to an emotion ("\underline{She gave me a gift}")
\end{itemize}

\pause
\vspace{0.5cm}

$\rightarrow$ Different emotion expression modes are more or less difficult to interpret %[add refs psycholinguistic, psychiatry, refs aline etienne]
\end{frame}

\begin{frame}{How to integrate psychological theories of emotion?}

%How to integrate psychological theories of emotion?

\pause
\vspace{0.5cm}

%I use the \textbf{integrated framework for emotion theories} (Scherer, 2022):

\begin{figure}
    \centering
    \includegraphics[width=0.9\linewidth]{img/scherer_integrated_framework.png}
    \caption{Integrated framework for emotion theories. Emotional episodes are synchronized changes in four components \citep{schererTheoryConvergenceEmotion2022a}.}
    \label{fig:placeholder}
\end{figure}

%[refs cortal et al.]
    
\end{frame}

\begin{frame}{}
\Large
\begin{center}
    Item modeling
    \section{Item modeling}
\end{center}

\end{frame}

\begin{frame}{Item modeling}

Music item models leverage collaborative and content data

    \vspace{0.25cm}
  \pause

  \begin{figure}
    \centering
    \includegraphics[scale=0.2]{deezer_img/onion_model.png}
    \caption{The \textit{onion model} of music content. A visualization of content layers accumulating over time, ranging from objective descriptors (inner) to high-level, subjective cultural data (outer).}
    \label{fig:user_modeling_diagram}
\end{figure}

\end{frame}

\begin{frame}{Item modeling}

  \textit{How can we construct language-based item representations that capture both intrinsic content and individual/cultural context?}

      \vspace{0.25cm}
  \pause

  We could use LLMs for several roles:

      \vspace{0.25cm}
  \pause

\begin{itemize}[<+->]
    \item LLMs as \textit{knowledge bases} to generate music tags, genres, or styles% without specific training
    \item LLMs as \textit{information extractors} to correct or augment metadata%, such as release years or artist details
    \item LLMs as \textit{summarizers} of heterogeneous textual data to create a concise summary of a song
\end{itemize}

\end{frame}

\begin{frame}{}
\Large
\begin{center}
    Music recommendation
    \section{Music recommendation}
\end{center}

\end{frame}

\begin{frame}{Music recommendation}

  \textit{How can we leverage natural language queries for context-aware music recommendation using LLMs?}

        \vspace{0.25cm}
  \pause

  \begin{itemize}[<+->]
    \item Users often employ broad concepts, such as \enquote{music for sports during a rainy day}, which implies a need to align music with specific contexts
    \item Given an user profile and a natural language query, how to recommend the best music items?
    \item Previous methods often rely on keyword matching, which limits the expressiveness of user intents
    \item LLMs can interpret complex natural language queries, enabling more nuanced recommendations
\end{itemize}

\end{frame}

\begin{frame}{Music recommendation}

\begin{figure}
    \centering
    \includegraphics[scale=0.25]{deezer_img/prompt_different_tasks.png}
    %\caption{User modeling diagram}
    \label{fig:user_modeling_diagram}
\end{figure}

\end{frame}

\begin{frame}{Music recommendation}

  \textit{How can we optimize prompts to improve recommendation quality?}

        \vspace{0.25cm}
  \pause

  \begin{itemize}[<+->]
    \item Prompt engineering is crucial for eliciting desired responses from LLMs
    \item Manual prompt design can be time-consuming and may not yield optimal results
    \item Automated prompt optimization techniques can explore the prompt space to find effective formulations
    \item We could improve recommender prompts by optimizing natural language instructions and few-shot examples
    \item We could improve LLM-as-a-judge prompts to better measure subjective dimensions of recommendation quality (\textit{e.g.}, groundedness, discovery quality, personalization gain, profile fidelity, and cultural/linguistic coverage)
\end{itemize}

\end{frame}

\begin{frame}{DSPy prompt optimization}

\begin{figure}
    \centering
    \includegraphics[scale=0.27]{deezer_img/dspy.png}
    %\caption{DSPy framework for prompt optimization.}
    \label{fig:dspy_prompt_optimization}
\end{figure}


\end{frame}

\begin{frame}{}
\Large
\begin{center}
    Conclusion
    \section{Conclusion}
\end{center}

\end{frame}

\begin{frame}{Conclusion}
  \begin{itemize}[<+->]
    \item The research proposal aims to develop language-based representations for music recommendation using LLMs
    \item Three interconnected axes: user modeling, item modeling, and music recommendation
    \item User modeling: \textit{How can we leverage LLMs to obtain scrutable natural language profiles? How can we better modelize user preferences, especially regarding their affective states?}
    \item Item modeling: \textit{How can we construct language-based item representations that capture both intrinsic content and individual/cultural context?}
    \item Music recommendation: \textit{How can we leverage natural language queries for context-aware music recommendation using LLMs? How can we optimize prompts to improve recommendation quality?}
  \end{itemize}
\end{frame}

\end{document}
