% --begin pap's style
\documentclass[handout,10pt]{beamer}

%\usetheme[secheader]{Darmstadt}
%\usetheme{Pittsburgh}

\usepackage[english]{babel}


\usepackage[utf8]{inputenc}
\usepackage[final]{pdfpages}
\usepackage[official]{eurosym}
\usepackage{graphicx}
\usepackage{caption}
\usepackage{csquotes}
\usepackage{multirow}
\usepackage{amsmath}
\usepackage{amsfonts} % for \text
\usepackage{hyperref}
\usepackage{comment}
\usepackage{graphicx}
\usepackage{booktabs}
\usepackage{tabularx}
\usepackage[flushleft]{threeparttable}  % for table notes
\newcommand\mscriptsize[1]{\mbox{\scriptsize\ensuremath{#1}}}
\newcommand\mtiny[1]{\mbox{\tiny\ensuremath{#1}}}

\usepackage{subcaption}

\usepackage[T1]{fontenc}
\usepackage[utf8]{inputenc}
\usepackage{minted}           % core package
\usepackage{xcolor}           % for background color
\definecolor{LightGray}{gray}{0.95}


\definecolor{blueNCS}{rgb}{0.0 0.53 0.74}  
\definecolor{bluepanam}{rgb}{0.0 0.189 0.79} 
\definecolor{Aplgreen}{rgb}{0.55 0.71 0.0}  
\definecolor{greentech}{cmyk}{0.7 0.0 1.0 0.0} 
\definecolor{DarkBlue}{rgb}{0.1,0.1,0.9}

\hypersetup{
    colorlinks=true,
    linkcolor=bluepanam,
    filecolor=bluepanam,      
    urlcolor=bluepanam,
    pdftitle={Overleaf Example},
    pdfpagemode=FullScreen,
    }

\setbeamercolor{palette primary}{bg=white,fg=black}
\setbeamercolor{palette secondary}{bg=white,fg=bluepanam}
\setbeamercolor{palette tertiary}{bg=white,fg=bluepanam}
\setbeamercolor{palette quaternary}{bg=white,fg=bluepanam}
\setbeamercolor{structure}{fg=bluepanam} % itemize, enumerate, etc
\setbeamercolor{section in toc}{fg=bluepanam} % TOC sections


%% \setbeamercolor{section in head/foot}{fg=white, bg=blue}
\setbeamercolor{title}{fg=bluepanam, bg=white}
\setbeamercolor{author}{fg=black, bg=white}
\setbeamercolor{institute}{fg=black, bg=white}
\setbeamercolor{date}{fg=white, bg=white}


\setbeamercolor{section in head/foot}{fg=bluepanam, bg=white}     
     
\setbeamercolor{author in head/foot}{fg=bluepanam, bg=white}
%\setbeamercolor{author in head/foot}{fg=white, bg=white}
\beamertemplatenavigationsymbolsempty
% --end pap's style

%\usepackage[backend=biber,
%            style=apa,
%            block=par,]{biblatex}
%\usepackage[backend=bibtex]{biblatex}
%\bibliography{sample.bib}
%\usepackage{natbib}
%\bibliographystyle{plain}  % or ieeetr, apalike, etc.
%\bibliography{sample_bibtex.bib}

%\newcommand{\customcite}[1]{\citeauthor{#1}, \citetitle{#1}, \citeyear{#1}, \citeurl{#1}}
%\newcommand{\customcitenourl}[1]{\citeauthor{#1}, \citetitle{#1}, \citeyear{#1}}
\newcommand{\parensmall}[1]{{\scriptsize #1}}

%\newcommand{\comment}[1]{}

\title{Natural language processing for subjectivity analysis in personal narratives}
\author{Gustave Cortal}
\date{\today}

\makeatletter
\defbeamertemplate*{footline}{myminiframes theme}
  {%
    \begin{beamercolorbox}[colsep=1.5pt]{upper separation line foot}
    \end{beamercolorbox}
    \begin{beamercolorbox}[ht=2.5ex,dp=1.125ex,%
      leftskip=.3cm,rightskip=.3cm plus1fil]{author in head/foot}%
      \leavevmode{\usebeamerfont{author in head/foot}}%
      \hfill%
    %\insertframenumber{}\,/\,\inserttotalframenumber%
    \end{beamercolorbox}%
    \begin{beamercolorbox}[ht=2.5ex,dp=2.125ex,leftskip=.3cm,rightskip=.3cm plus1fil]{section in head/foot}%
      %%      {\usebeamerfont{section in head/foot} somthing written here \hfill  \setlength{\fboxrule}{0pt}\setlength{\fboxsep}{0pt}\fcolorbox{blueNCS}{blueNCS!70}{My own image here}}%
      {\usebeamerfont{section in head/foot} \insertshortauthor \hfill  %\setlength{\fboxrule}{0pt}\setlength{\fboxsep}{0pt}\fcolorbox{blueNCS}{blueNCS!100}{foobar etc}
      }%
      \insertframenumber{}\,/\,\inserttotalframenumber%
    \end{beamercolorbox}%
    \begin{beamercolorbox}[colsep=1.5pt]{lower separation line foot}
    \end{beamercolorbox}
  }
  \makeatother


\begin{document}

\setlength{\parskip}{5pt}%
\setlength{\parsep}{0pt}%
\setlength{\itemsep}{0.25cm}%
\setlength{\leftmargini}{0.5cm}

\begin{frame}
  \titlepage
  \vspace{-1.5cm}
  \begin{center}
    \includegraphics[height=1.5cm]{img/logo_ens_saclay.png}
  \end{center}
\end{frame}


\begin{frame}{Introduction}

%\pause

%\textbf{Observation}: The field emphasizes formal and mathematical tasks, whereas social and emotional tasks remains underexplored

%\textbf{Observation}: LLMs have mastered linguistic knowledge but still lack functional skills such as emotional reasoning %The research field focuses more on formal reasoning compared to emotional and social reasoning.
%contastat de these au debut, moins vrai mtn

%\vspace{0.5cm}
\pause

\textbf{Research question}: How to model subjective experience in narratives?


\vspace{0.5cm}
\pause

\textbf{Steps}:

\begin{itemize}[<+->]
    \item Definition of objectives and scope using cognitive science
    \item Construction of an emotion dataset 
    \item Training of language models for emotion analysis 
    \item Formalization of style in narratives 
\end{itemize}

\vspace{0.5cm}
\pause

\small

\textit{Each step lead to a first-author article in an international conference} %(including EMNLP and LREC-COLING)

%\textit{My research models are publicly hosted on Hugging Face and were trained using the Jean Zay supercomputer}
    
\end{frame}

\begin{frame}{Definition of objectives and scope using cognitive science}

\textbf{Goal}: Identify limitations and research directions

\pause
\vspace{0.5cm}

I review psychological theories of emotion and emotion annotation schemes in NLP

\pause
\vspace{0.5cm}

What are current \textbf{limitations}? 

\begin{itemize}[<+->]
    \item Different emotion theories lead to divergences in how to annotate them in the text
    \item Some linguistic and cognitive science theories are not considered
    \item There is no benchmark that evaluates the richness of the emotional phenomenon
\end{itemize}


\vspace{0.5cm}

\scriptsize

\textbf{G. Cortal} and C. Bonard. \href{https://aclanthology.org/2024.cmcl-1.23/}{Improving Language Models for Emotion Analysis: Insights from Cognitive Science}. \textit{CMCL, ACL 2024}.
    
\end{frame}

\begin{frame}{Definition of objectives and scope using cognitive science}

How to integrate psychological theories of emotion?

\pause
\vspace{0.5cm}

I use the integrated framework for emotion theories (Scherer, 2022):

\begin{figure}
    \centering
    \includegraphics[width=0.8\linewidth]{img/scherer_integrated_framework.png}
    \caption{Emotional episodes are synchronized changes in four components.}
    \label{fig:placeholder}
\end{figure}
    
\end{frame}

\begin{frame}{Construction of an emotion dataset}

\textbf{Goal}: A more comprehensive understanding of emotional events

\pause

\begin{table}
    \centering
    \resizebox{0.9\textwidth}{!}{
\begin{tabular}{l|p{0.78\textwidth}}
 
                   \textbf{Component} &
                 \textbf{Answer} \\
 
\hline
          \textsc{behavior} & I'm giving a lecture on a Friday morning at 8:30. A student goes out and comes back a few moments later with a coffee in his hand. \\
\textsc{feeling} & My heart is beating fast, and I freeze, waiting to know how to act. \\
  \textsc{thinking} & I think this student is disrupting my class. \\
\textsc{territory} & The student attacks my ability to be respected in class. \\
 
\end{tabular}}
%\captionof{table}{Example of an emotional narrative structured according to emotion components. More than 1,000 narratives were collected using emotion regulation questionnaires.}
\label{tab:description_corpus}
\end{table} % open-ended questions, my thesis director and I

\small
More than 1,000 narratives were collected during emotion regulation sessions
\vspace{0.5cm}

\scriptsize

\textbf{G. Cortal}, A. Finkel, P. Paroubek, L. Ye. \href{https://aclanthology.org/2023.latechclfl-1.8/}{Emotion Recognition based on Psychological Components in Guided Narratives for Emotion Regulation}. \textit{SIGHUM, EACL 2023}.% \href{https://underline.io/lecture/71953-emotion-recognition-based-on-psychological-components-in-guided-narratives-for-emotion-regulation}{video}.
    
\end{frame}

\begin{frame}{Training language models for emotion analysis}

\textbf{Goal}: Discrete emotion detection based on components

\vspace{0.5cm}
\pause

\begin{table}
    \centering
    \resizebox{0.9\textwidth}{!}{
\begin{tabular}{l|lllllll}

&\multicolumn{3}{c}{\textbf{Logistic Regression}}&\multicolumn{3}{c}{\textbf{CamemBERT}} \\

                   \textbf{Component} &  Precision &     Recall &   $F_1$ &  Precision &     Recall &   $F_1$ \\
\hline
              All  & 71.2\,\mscriptsize{(2.6)} & 69.1\,\mscriptsize{(2.2)} & 67.8\,\mscriptsize{(2.3)} & \textbf{85.1} & \textbf{84.8} & \textbf{84.7} \\
              Without \textsc{behavior}   & 77.4\,\mscriptsize{(2.3)} & 75.8\,\mscriptsize{(2.4)} & 74.5\,\mscriptsize{(2.6)} & 80.3 & 79.8 & 79.7 \\
              Without \textsc{feeling}  & 64.3\,\mscriptsize{(1.9)} & 61.5\,\mscriptsize{(1.2)} & 61.3\,\mscriptsize{(2.2)} & 81.6 & 79.8 & 79.9  \\
              Without \textsc{thinking}  & 70.9\,\mscriptsize{(1.8)} & 69.1\,\mscriptsize{(2.0)} & 68.3\,\mscriptsize{(2.2)} & 79.6 & 78.5 & 78.7 \\
              Without \textsc{territory}  & 64.3\,\mscriptsize{(4.1)} & 64.5\,\mscriptsize{(2.4)} & 62.3\,\mscriptsize{(2.8)} & 78.7 & 78.5 & 78.6 \\
          Only \textsc{behavior}  & 52.1\,\mscriptsize{(3.5)} & 54.6\,\mscriptsize{(2.9)} & 51.7\,\mscriptsize{(2.9)} & 68.4  & 67.1 & 66.6 \\
Only \textsc{feeling}  & 69.6\,\mscriptsize{(1.5)} & 68.9\,\mscriptsize{(2.1)} & 68.4\,\mscriptsize{(2.0)} & 67.8 & 68.4 & 67.7 \\
  Only \textsc{thinking}  & 50.1\,\mscriptsize{(3.4)} & 53.8\,\mscriptsize{(2.3)} & 50.6\,\mscriptsize{(2.7)} & 70.5 & 70.1 & 70.1 \\
              Only \textsc{territory}  & 68.2\,\mscriptsize{(1.8)} & 66.8\,\mscriptsize{(2.2)} & 66.6\,\mscriptsize{(2.3)} & 71.4 & 68.4 & 68.9 \\
\end{tabular}}
    %\caption{Scores (± std) for discrete emotion classification based on components.}
    \label{tab:pred_emotion}
\end{table}

\vspace{0.5cm}

\scriptsize

\textbf{G. Cortal}, A. Finkel, P. Paroubek, L. Ye. \href{https://aclanthology.org/2023.latechclfl-1.8/}{Emotion Recognition based on Psychological Components in Guided Narratives for Emotion Regulation}. \textit{SIGHUM, EACL 2023}.
    
\end{frame}

\begin{frame}{Training language models for emotion analysis}

Need other datasets with narrative structure, emotional content, and available for research

\vspace{0.5cm}
\pause

Quantitative dream analysis examines recurring patterns between narrative elements using a database of dream narratives and an annotation scheme (Domhoff, 2004)

\vspace{0.5cm}
\pause

The annotation process is complex and costly

\vspace{0.5cm}
\pause

How to automate the annotation process?

%side project in my PhD
    
\end{frame}

\begin{frame}{Training language models for emotion analysis}

\textbf{Goal}: Character and emotion detection in dream narratives

\pause

\begin{figure}
    \centering
    \includegraphics[width=0.6\linewidth]{img/dream_method.png}
    %\caption{Codes describing characters and their emotions are converted into
%natural language to produce the training data.}
    \label{fig:placeholder}
\end{figure}

\scriptsize

\textbf{G. Cortal}. \href{https://aclanthology.org/2024.lrec-main.1282/}{Sequence-to-Sequence Language Models for Character and Emotion Detection in Dream Narratives}. \textit{LREC-COLING 2024}.
    
\end{frame}

\begin{frame}{Training language models for emotion analysis}

\scriptsize

LaMini-Flan-T5 finetuned on 1823 dream narratives

\small

\begin{table}
    \centering
    \resizebox{0.9\textwidth}{!}{
\begin{tabular}{l|cccccc}
\textbf{Model} & \textbf{Status} & \textbf{Gender} & \textbf{Identity} & \textbf{Age} & \textbf{Character} & \textbf{Emotion} \\
\hline
\textsc{baseline} & 82.87 & 78.02 & 76.17 & 86.21 & 64.74 &  75.13 \\
\hline
\textsc{no\textsubscript{semantics}} & 71.37 & 56.54\textbf{*} & 61.0  & 90.51  & 41.79\textbf{*} &  75.79 \\
\textsc{no\textsubscript{names}} & 80.66\textbf{*}  & 74.32\textbf{**}  & 74.2  & 83.95\textbf{*}  & 60.93\textbf{**} &  73.04\textbf{*} \\
\hline
\textsc{size\textsubscript{small}} & 78.35\textbf{**}  & 72.13\textbf{**}  & 70.25\textbf{**}  & 81.66\textbf{**}  & 56.79\textbf{**} &  70.15\textbf{**} \\
\textsc{size\textsubscript{large}} & 84.51\textbf{*} & 80.3\textbf{**}  & 78.63\textbf{**}  & 87.29  & 67.63\textbf{**} & 74.71 \\
\hline
\textsc{first\textsubscript{group}} & 82.33  & 77.71  & 74.86  & 85.61  & 63.71 &  71.94 \\
\textsc{first\textsubscript{individual}} & 80.59\textbf{**}  & 76.14  & 74.22\textbf{*}  & 83.87\textbf{**} & 62.67 &  67.32 \\
\textsc{first\textsubscript{emotion}} & 83.92 & 78.74 & 77.06  & 87.63  & 64.97 &  72.03 \\
\hline
\textsc{conversion\textsubscript{comma}} & 84.02\textbf{**}  & 79.84\textbf{**}  & 77.67\textbf{**}  & 87.08\textbf{*}  & 66.69\textbf{**} &  73.68 \\
\textsc{conversion\textsubscript{marker}} & 82.39  & 78.45  & 76.53  & 86.09  & 65.44 &  74.36 \\
\hline
\textsc{StableBeluga\textsubscript{1}} &   43.95\textbf{**} &   39.76\textbf{**} &     31.25\textbf{**} &  56.16\textbf{**} &      15.65\textbf{**} & -\\
\textsc{StableBeluga\textsubscript{3}} &   52.44\textbf{**} &   46.49\textbf{**} &     38.46\textbf{**} &  63.88\textbf{**} &      21.06\textbf{**} & -\\
\textsc{StableBeluga\textsubscript{5}} &   55.89\textbf{**} &   46.29\textbf{**} &     42.61\textbf{**} &  63.73\textbf{**} &      24.86\textbf{**} & - \\
\hline
\textsc{cross-validation} & 86.28 & 81.9  & 79.51  & 89.52 & 68.64 &  76.18\\
\end{tabular}}
    \label{tab:result}
\end{table}

\vspace{0.5cm}
\scriptsize

%Generated dream narratives available on \href{https://gustavecortal.com/project/oneirogen}{gustavecortal.com}.

%\href{https://huggingface.co/gustavecortal/oneirogen-7B}{Oneirogen}, a language model for dream generation. 2024.

%\href{https://huggingface.co/gustavecortal/dream-t5}{Dream‑T5}, a language model for emotion and character prediction in dream narratives. 2023.

\textbf{G. Cortal}. \href{https://aclanthology.org/2024.lrec-main.1282/}{Sequence-to-Sequence Language Models for Character and Emotion Detection in Dream Narratives}. \textit{LREC-COLING 2024}.
    
\end{frame}



\begin{frame}{Formalization of style in narratives}

\textit{How} subjective experience is communicated?

\vspace{0.5cm}
\pause

\textbf{Goal}: Formalize style as patterns of linguistic choices that encode subjective experience

\pause
\vspace{0.5cm}

\begin{table}[!ht]
  \centering
  \small
  \renewcommand{\arraystretch}{1.1}
  \begin{threeparttable}
  
    \begin{tabular}{l|l|l}
      \textbf{Clause} & \textbf{Process (code)} & \textbf{Participants} \\
      \midrule
      I wake in a dark room         & Action (\textbf{a})  & Actor \\
      I feel a cold wind            & Mental (\textbf{m})  & Senser,\\
                                            &             & Phenomenon \\
      I tell myself to move         & Verbal (\textbf{v})  & Sayer,\\
                                            &             & Recipient \\
      \bottomrule
    \end{tabular}

    \begin{tablenotes}[flushleft]
      \footnotesize
      \item \textbf{Sequence:} $amv$\quad|\quad
            \textbf{Substrings:} \{am, mv\}
    \end{tablenotes}
  \end{threeparttable}
  \caption{Illustrative pipeline. Each narrative is mapped to a symbolic sequence using an alphabet based on extracted features}
    \label{tab:example}
\end{table}

\scriptsize

\textbf{G. Cortal} and A. Finkel. \href{https://gustavecortal.com/data/Formalizing_Style_in_Personal_Narratives.pdf}{Formalizing Style in Personal Narratives}. \textit{EMNLP 2025}.
    
\end{frame}

\begin{frame}{Formalization of style in narratives}

\begin{figure}
    \centering
    \includegraphics[width=0.8\linewidth]{img/blind_odds_3.png}
    \caption{Top substring odds ratio between the blind series and the norm. Based on 381 sequences from 15 blind dreamers}
    \label{fig:placeholder}
\end{figure}

\vspace{0.5cm}

\scriptsize

\textbf{G. Cortal} and A. Finkel. \href{https://gustavecortal.com/data/Formalizing_Style_in_Personal_Narratives.pdf}{Formalizing Style in Personal Narratives}. \textit{EMNLP 2025}.
    
\end{frame}

\begin{frame}{Conclusion}

\textbf{Research question}: How to model subjective experience in narratives?

\vspace{0.5cm}
\pause

\textbf{Steps}:

\begin{itemize}[<+->]
    \item Definition of objectives and scope using cognitive science
    \item Construction of an emotion dataset 
    \item Training of language models for emotion analysis 
    \item Formalization of style in narratives
\end{itemize}

\vspace{0.5cm}
%\pause

\small

\textit{My research models are publicly hosted on Hugging Face and were trained using the Jean Zay supercomputer}
    
\end{frame}

\begin{frame}{Appendix}

\begin{itemize}
    \item \textit{Fine-grained mental health topic modeling in different populations using large language models} (PhD internship, to be submitted for Nature Mental Health)
    \item \href{https://huggingface.co/gustavecortal/Piaget-8B}{Piaget}, a language model for psychological and philosophical reasoning, 2025.
    \item N. Richet, S. Belharbi, H. Aslam, M. Schadt, M. González-González, \textbf{G. Cortal}, A. Koerich, M. Pedersoli, A. Finkel, S. Bacon, E. Granger. \href{https://arxiv.org/abs/2407.12927}{Textualized and Feature-based Models for Compound Multimodal Emotion Recognition in the Wild}. \textit{ABAW, ECCV 2024}.
\end{itemize}
    
\end{frame}

\end{document}
