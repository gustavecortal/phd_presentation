% --begin pap's style
\documentclass[handout,10pt]{beamer}

%\usetheme[secheader]{Darmstadt}
%\usetheme{Pittsburgh}

\usepackage[english]{babel}


\usepackage[utf8]{inputenc}
\usepackage[final]{pdfpages}
\usepackage[official]{eurosym}
\usepackage{graphicx}
\usepackage{caption}
\usepackage{csquotes}
\usepackage{multirow}
\usepackage{amsmath}
\usepackage{amsfonts} % for \text
\usepackage{hyperref}
\usepackage{comment}
\usepackage{graphicx}
\usepackage{booktabs}
\usepackage{tabularx}
\usepackage[flushleft]{threeparttable}  % for table notes
\newcommand\mscriptsize[1]{\mbox{\scriptsize\ensuremath{#1}}}
\newcommand\mtiny[1]{\mbox{\tiny\ensuremath{#1}}}
\usepackage{xurl}             % allow line breaks in long URLs
\usepackage{ragged2e}         % better ragged-right with hyphenation
\usepackage{xspace}           % for intelligent spacing after commands
\setlength{\emergencystretch}{3em} % gentle last-resort stretch

\usepackage{subcaption}

\usepackage[T1]{fontenc}
\usepackage[utf8]{inputenc}
%\usepackage{minted}           % core package
\usepackage{xcolor}           % for background color
\definecolor{LightGray}{gray}{0.95}


\definecolor{blueNCS}{rgb}{0.0, 0.53, 0.74}    % ✓ Commas
\definecolor{bluepanam}{rgb}{0.0, 0.189, 0.79} % ✓ Commas
\definecolor{Aplgreen}{rgb}{0.55, 0.71, 0.0}   % ✓ Commas
\definecolor{DarkBlue}{rgb}{0.1,0.1,0.9}

\hypersetup{
    colorlinks=false,
    linkcolor=bluepanam,
    filecolor=bluepanam,      
    urlcolor=bluepanam,
    pdftitle={Overleaf Example},
    pdfpagemode=FullScreen,
    }

\setbeamercolor{palette primary}{bg=white,fg=black}
\setbeamercolor{palette secondary}{bg=white,fg=bluepanam}
\setbeamercolor{palette tertiary}{bg=white,fg=bluepanam}
\setbeamercolor{palette quaternary}{bg=white,fg=bluepanam}
\setbeamercolor{structure}{fg=bluepanam} % itemize, enumerate, etc
\setbeamercolor{section in toc}{fg=bluepanam} % TOC sections


%% \setbeamercolor{section in head/foot}{fg=white, bg=blue}
\setbeamercolor{title}{fg=bluepanam, bg=white}
\setbeamercolor{author}{fg=black, bg=white}
\setbeamercolor{institute}{fg=black, bg=white}
\setbeamercolor{date}{fg=white, bg=white}


\setbeamercolor{section in head/foot}{fg=bluepanam, bg=white}     
     
\setbeamercolor{author in head/foot}{fg=bluepanam, bg=white}
%\setbeamercolor{author in head/foot}{fg=white, bg=white}
\beamertemplatenavigationsymbolsempty
% --end pap's style

%\newcommand{\customcite}[1]{\citeauthor{#1}, \citetitle{#1}, \citeyear{#1}, \citeurl{#1}}
%\newcommand{\customcitenourl}[1]{\citeauthor{#1}, \citetitle{#1}, \citeyear{#1}}
\newcommand{\parensmall}[1]{{\scriptsize #1}}

% Custom fullcite command without URL, DOI, and urldate
\newcommand{\fullcitenourl}[1]{%
  \AtNextCite{%
    \clearfield{url}%
    \clearfield{doi}%
    \clearfield{urldate}%
  }%
  \fullcite{#1}%
}


\usepackage{xurl}             % allow line breaks in long URLs
\usepackage{ragged2e}         % better ragged-right with hyphenation
\usepackage{xspace}           % for intelligent spacing after commands
\setlength{\emergencystretch}{3em} % gentle last-resort stretch

%\newcommand{\comment}[1]{}

\usepackage[style=authoryear,natbib=true,backend=biber,date=year,urldate=long]{biblatex}
\addbibresource{phd_slides_biblatex.bib}

% Remove "visited on" dates from URLs
\AtEveryBibitem{\clearfield{urldate}}
\hypersetup{
  colorlinks=true,
  linkcolor=bluepanam,
  filecolor=bluepanam,      
  urlcolor=bluepanam,
  citecolor=black,  % Changed to black to match text color
  pdftitle={Overleaf Example},
  pdfpagemode=FullScreen,
}

% Add more spacing between entries
\setlength\bibitemsep{0.5\baselineskip}

\title{Natural language processing for\\ subjectivity analysis in personal narratives}
\author{Gustave Cortal}
\institute{\footnotesize Thesis director: Alain Finkel \\ Co-advisors: Patrick PAROUBEK and Lina YE}
\titlegraphic{
  \includegraphics[width=4cm]{img/lmf_logo_emnlp.png}
  \hspace{1cm}
  \includegraphics[width=5cm]{img/logo_ens_saclay.png}
}
\date{\today}

\makeatletter
\defbeamertemplate*{footline}{myminiframes theme}
  {%
    \begin{beamercolorbox}[colsep=1.5pt]{upper separation line foot}
    \end{beamercolorbox}
    \begin{beamercolorbox}[ht=2.5ex,dp=1.125ex,%
      leftskip=.3cm,rightskip=.3cm plus1fil]{author in head/foot}%
      \leavevmode{\usebeamerfont{author in head/foot}}%
      \hfill%
    %\insertframenumber{}\,/\,\inserttotalframenumber%
    \end{beamercolorbox}%
    \begin{beamercolorbox}[ht=2.5ex,dp=2.125ex,leftskip=.3cm,rightskip=.3cm plus1fil]{section in head/foot}%
      %%      {\usebeamerfont{section in head/foot} somthing written here \hfill  \setlength{\fboxrule}{0pt}\setlength{\fboxsep}{0pt}\fcolorbox{blueNCS}{blueNCS!70}{My own image here}}%
      {\usebeamerfont{section in head/foot} \insertshortauthor \hfill  %\setlength{\fboxrule}{0pt}\setlength{\fboxsep}{0pt}\fcolorbox{blueNCS}{blueNCS!100}{foobar etc}
      }%
      \insertframenumber{}\,/\,\inserttotalframenumber%
    \end{beamercolorbox}%
    \begin{beamercolorbox}[colsep=1.5pt]{lower separation line foot}
    \end{beamercolorbox}
  }
  \makeatother


\begin{document}

\setlength{\parskip}{5pt}%
\setlength{\parsep}{0pt}%
\setlength{\itemsep}{0.25cm}%
\setlength{\leftmargini}{0.5cm}

\begin{frame}
  \titlepage
\end{frame}

\begin{frame}{}
\Large
\begin{center}
    Introduction
    \section{Introduction}
\end{center}
\end{frame}


\begin{frame}{Context}

  \begin{itemize}[<+->]
    \item Natural language processing for psychology is underexplored%, despite its fundamental importance for understanding human language,
    \item We build on an existing subfield: emotion analysis
    \item We study subjectivity (involving first-person perspective, meaning-making processes, and experiential content)
    \item We focus on personal narratives (\textit{e.g.}, emotional narratives, dream reports, mental health narratives)
    %\item Computational approaches to analyzing subjective experience have potential for supporting psychological well-being
    \end{itemize}

    \vspace{0.5cm}
    \pause

    We first address the \textit{content} by classifying elements of personal narratives (\textit{e.g.}, characters and emotions). Then, we study the \textit{form} through the concept of style

\end{frame}


\begin{frame}{Introduction}

How to model subjective experience in personal narratives?

\vspace{0.5cm}
\pause

\begin{itemize}[<+->]
    \item Definition of objectives and scope using cognitive science
    \item Construction of an emotion dataset 
    \item Training of language models for emotion analysis 
    \item Formalization of style in personal narratives
    %\item Automatic thematic analysis in mental health narratives %using language models
\end{itemize}

\vspace{0.5cm}
\pause

\small

\textit{My research models are publicly hosted on Hugging Face and were trained using the Jean Zay supercomputer}
    
\end{frame}

\begin{frame}{}
\Large
\begin{center}
    Definition of objectives using cognitive science
    \section{Definition of objectives using cognitive science}
\end{center}
\vspace{1.5cm}

    \footnotesize

    \textbf{G. Cortal} and C. Bonard. \href{https://aclanthology.org/2024.cmcl-1.23/}{Improving Language Models for Emotion Analysis: Insights from Cognitive Science}. \textit{CMCL, ACL 2024}.
\end{frame}

\begin{frame}{Psychological theories and emotion annotation schemes}

What are current limitations and interesting research directions?

%\pause
%\vspace{0.25cm}

%We review psychological theories of emotion and emotion annotation schemes in NLP

\pause
\vspace{0.25cm}

%What are current limitations? 

%[add refs to each theory and annotation schemes]

\begin{center}
\begin{tabular}{p{6.em}|p{10.em}|p{12.em}}
Psychological theories & In text, emotion is... & Example \\
\hline
Basic emotions theory & a \textbf{category} & "I love philosophy." $\rightarrow$ \texttt{joy} \\
Constructivist theories & a continuous value with an \textbf{affective} meaning & "His voice soothes me." $\rightarrow$ \texttt{valence} (4/5), \texttt{arousal} (1/5) \\
Appraisal theory & a continuous value with a \textbf{cognitive} meaning & "I received a surprise gift." $\rightarrow$ \texttt{sudden} (4/5), \texttt{control} (0/5) \\ 
& composed of \textbf{semantic roles} & "Louise (\texttt{experiencer}) was angry (\texttt{cue}) towards Paul (\texttt{target}), because he didn’t inform her (\texttt{cause})."
\end{tabular}
\end{center}


%\vspace{0.5cm}

%\scriptsize

%\textbf{G. Cortal} and C. Bonard. \href{https://aclanthology.org/2024.cmcl-1.23/}{Improving Language Models for Emotion Analysis: Insights from Cognitive Science}. \textit{CMCL, ACL 2024}.
    
\end{frame}

\begin{frame}{Limitations in emotion analysis}

\begin{itemize}[<+->]
    \item Different emotion theories lead to divergences in how to annotate them in the text
    \item Some linguistic and cognitive science theories are not considered
    \item There is no benchmark that evaluates the richness of the emotional phenomenon
\end{itemize}

%\cite{ekman_basic_1999}, \cite{russell_circumplex_1980,barrett_how_2017}, \cite{arnold_emotion_1960}

\end{frame}

\begin{frame}{}
\Large
\begin{center}
    Linguistic and cognitive science theories
    %\section{Definition of objectives using cognitive science}
\end{center}

\end{frame}

\begin{frame}{Which verbal signs are used to infer expressed emotions?}

%Which verbal signs are used to infer expressed emotions?

%\pause
%\vspace{0.5cm}

Raphaël Micheli categorizes a range of linguistic markers into three \textit{emotion expression modes} \citep{micheliEsquisseDuneTypologie2013}. The emotion can be: 

%\cite{Micheli2014}
\pause
\vspace{0.5cm}

\begin{itemize}[<+->]
    \item \textit{labeled} explicitly with an emotional term ("I am \underline{sad}")
    \item \textit{shown} with utterance features such as interjections and punctuations ("\underline{Ah!} That's great\underline{!}")
    \item \textit{suggested} with the description of a situation which generally, in a given sociocultural context, leads to an emotion ("\underline{She gave me a gift}")
\end{itemize}

\pause
\vspace{0.5cm}

$\rightarrow$ Different emotion expression modes are more or less difficult to interpret %[add refs psycholinguistic, psychiatry, refs aline etienne]
\end{frame}

\begin{comment}


\begin{frame}{What are the psychological mechanisms used to infer what is communicated?}

A \textit{code} is a pre-established pairing between stimuli and sets of information

\pause
\vspace{0.5cm}

The Morse code is a pairing between <combination of short and long signals> and [letters]% that senders and receivers must share to communicate with it. 

\pause
\vspace{0.5cm}

The formal semantics of a language is made of syntactical and lexical rules that pairs <strings of words> with [sentential meanings] %\cite{heim_semantics_1998}

%What are the psychological mechanisms used to infer what is communicated? %\cite{grice_logic_1975}


\end{frame}

\begin{frame}{What are the psychological mechanisms used to infer what is communicated?}

\begin{figure}
    \centering
    \includegraphics[scale=0.20]{img/scherer_dictionary_eng.png}
    \caption{Dictionary analysis in cognitive pragmatics.}% [cite]}
\end{figure}


\end{frame}

\begin{frame}{Codes underdetermine emotion meaning}

\pause

Let's take emotion expression modes as an example:

\vspace{0.5cm}
\pause

\begin{itemize}[<+->]
  \item \textit{Labeled}: \enquote{I am happy now} is explicit about the feeling but does not encode what the emotion is about
  \item \textit{Displayed}: interjections (\enquote{Wow!}, \enquote{Ah!}, \enquote{Damn!}) show affect yet leave valence and focus unclear
  \item \textit{Suggested}: \enquote{The ship has black sails.} can communicate any kind of emotion
\end{itemize}

\vspace{0.5cm}
\pause


$\rightarrow$ We rely on other sources of evidence to infer what is communicated

\end{frame}

\begin{frame}{What are the psychological mechanisms used to infer what is communicated?}

%What are the psychological mechanisms used to infer what is communicated? %\cite{grice_logic_1975}

%\pause
%\vspace{0.5cm}

\begin{figure}
    \centering
    \includegraphics[scale=0.20]{img/scherer_dictionary_eng.png}
    \caption{Dictionary analysis in cognitive pragmatics.} %[cite]}
\end{figure}


\end{frame}

\end{comment}

\begin{frame}{What are the psychological mechanisms used to infer what is communicated?}

%What are the psychological mechanisms used to infer what is communicated? %\cite{grice_logic_1975}

%\pause
%\vspace{0.5cm}

\begin{figure}
    \centering
    \includegraphics[scale=0.20]{img/scherer_eng.png}
    \caption{Detective analysis in cognitive pragmatics.} %[cite]}
\end{figure}


\end{frame}


\begin{frame}{How to integrate psychological theories of emotion?}

%How to integrate psychological theories of emotion?

\pause
\vspace{0.5cm}

%I use the \textbf{integrated framework for emotion theories} (Scherer, 2022):

\begin{figure}
    \centering
    \includegraphics[width=0.9\linewidth]{img/scherer_integrated_framework.png}
    \caption{Integrated framework for emotion theories. Emotional episodes are synchronized changes in four components \citep{schererTheoryConvergenceEmotion2022a}.}
    \label{fig:placeholder}
\end{figure}

%[refs cortal et al.]
    
\end{frame}

\begin{comment}

  \begin{frame}{Research directions}
  \begin{itemize}
    \item Construct a \textbf{unified annotation scheme} to capture the emotional phenomenon better and benefit from knowledge transfer between tasks
    \item Build \textbf{benchmarks} that evaluate various aspects of the emotional phenomenon [based on human difficulty]% based on emotion expression modes in linguistics and detective analysis in pragmatics
    \item Develop new \textbf{methods} in natural language processing [be more precise]% based on detective analysis to improve performance and explainability of models
\end{itemize}
\end{frame}

\end{comment}

\begin{frame}{}
\Large
\begin{center}
    Construction of an emotion dataset
    \section{Construction of an emotion dataset}

    \vspace{1.5cm}
\end{center}

\small
Available at \href{https://huggingface.co/datasets/gustavecortal/FrenchEmotionalNarratives}{hf.co/datasets/gustavecortal/FrenchEmotionalNarratives}

\vspace{0.5cm}

\textbf{G. Cortal}, A. Finkel, P. Paroubek, L. Ye. \href{https://aclanthology.org/2023.latechclfl-1.8/}{Emotion Recognition based on Psychological Components in Guided Narratives for Emotion Regulation}. \textit{SIGHUM, EACL 2023}.% \href{https://underline.io/lecture/71953-emotion-recognition-based-on-psychological-components-in-guided-narratives-for-emotion-regulation}{video}.
\end{frame}

\begin{frame}{French emotional narratives based on components}

\textbf{Goal}: A more comprehensive understanding of emotional events

\vspace{0.25cm}
\pause

\begin{table}
    \centering
    \resizebox{1\textwidth}{!}{
\begin{tabular}{l|p{0.78\textwidth}}
 
                   \textbf{Component} &
                 \textbf{Answer} \\
 
\hline
          \texttt{Behavior} & I'm giving a lecture on a Friday morning at 8:30. A student goes out and comes back a few moments later with a coffee in his hand. \\
\texttt{Feeling} & My heart is beating fast, and I freeze, waiting to know how to act. \\
  \texttt{Thinking} & I think this student is disrupting my class. \\
\texttt{Territory} & The student attacks my ability to be respected in class. \\

\end{tabular}}
%\captionof{table}{Example of an emotional narrative structured according to emotion components. More than 1,000 narratives were collected using emotion regulation questionnaires.}
\label{tab:description_corpus}
\end{table} % open-ended questions, my thesis director and I

\small
\vspace{0.25cm}

Chosen emotion: \texttt{anger} (possible choices: \texttt{anger}, \texttt{fear}, \texttt{joy}, \texttt{sadness})



\vspace{0.5cm}
\small
More than 1,000 narratives were collected during emotion regulation sessions
%\vspace{0.5cm}

%\scriptsize

%\textbf{G. Cortal}, A. Finkel, P. Paroubek, L. Ye. \href{https://aclanthology.org/2023.latechclfl-1.8/}{Emotion Recognition based on Psychological Components in Guided Narratives for Emotion Regulation}. \textit{SIGHUM, EACL 2023}.% \href{https://underline.io/lecture/71953-emotion-recognition-based-on-psychological-components-in-guided-narratives-for-emotion-regulation}{video}.
    
\end{frame}

\begin{frame}{}
\Large
\begin{center}
    Training language models for emotion analysis
    \section{Training language models for emotion analysis}
\end{center}

\vspace{1.5cm}

\footnotesize

%Generated dream narratives available on \href{https://gustavecortal.com/project/oneirogen}{gustavecortal.com}.

%\href{https://huggingface.co/gustavecortal/oneirogen-7B}{Oneirogen}, a language model for dream generation along with \href{https://huggingface.co/datasets/gustavecortal/DreamBank-annotated}{27,000 annotated dream narratives}

%\href{https://huggingface.co/gustavecortal/dream-t5}{Dream-T5}, a language model for emotion and character prediction in dream narratives

\textbf{G. Cortal}, A. Finkel, P. Paroubek, L. Ye. \href{https://aclanthology.org/2023.latechclfl-1.8/}{Emotion Recognition based on Psychological Components in Guided Narratives for Emotion Regulation}. \textit{SIGHUM, EACL 2023}

\vspace{0.5cm}

\textbf{G. Cortal}. \href{https://aclanthology.org/2024.lrec-main.1282/}{Sequence-to-Sequence Language Models for Character and Emotion Detection in Dream Narratives}. \textit{LREC-COLING 2024}

\end{frame}

\begin{frame}{Discrete emotion detection based on components}

%\textbf{Goal}: Discrete emotion detection based on components

%\vspace{0.5cm}
%\pause

\begin{table}
    \centering
    \resizebox{1\textwidth}{!}{
\begin{tabular}{l|lllllll}

&\multicolumn{3}{c}{\textbf{Logistic Regression}}&\multicolumn{3}{c}{\textbf{CamemBERT}} \\

                   \textbf{Component} &  Precision &     Recall &   $F_1$ &  Precision &     Recall &   $F_1$ \\
\hline
              All  & 71.2\,\mscriptsize{(2.6)} & 69.1\,\mscriptsize{(2.2)} & 67.8\,\mscriptsize{(2.3)} & \textbf{85.1} & \textbf{84.8} & \textbf{84.7} \\
              Without \texttt{behavior}   & 77.4\,\mscriptsize{(2.3)} & 75.8\,\mscriptsize{(2.4)} & 74.5\,\mscriptsize{(2.6)} & 80.3 & 79.8 & 79.7 \\
              Without \texttt{feeling}  & 64.3\,\mscriptsize{(1.9)} & 61.5\,\mscriptsize{(1.2)} & 61.3\,\mscriptsize{(2.2)} & 81.6 & 79.8 & 79.9  \\
              Without \texttt{thinking}  & 70.9\,\mscriptsize{(1.8)} & 69.1\,\mscriptsize{(2.0)} & 68.3\,\mscriptsize{(2.2)} & 79.6 & 78.5 & 78.7 \\
              Without \texttt{territory}  & 64.3\,\mscriptsize{(4.1)} & 64.5\,\mscriptsize{(2.4)} & 62.3\,\mscriptsize{(2.8)} & 78.7 & 78.5 & 78.6 \\
          Only \texttt{behavior}  & 52.1\,\mscriptsize{(3.5)} & 54.6\,\mscriptsize{(2.9)} & 51.7\,\mscriptsize{(2.9)} & 68.4  & 67.1 & 66.6 \\
Only \texttt{feeling}  & 69.6\,\mscriptsize{(1.5)} & 68.9\,\mscriptsize{(2.1)} & 68.4\,\mscriptsize{(2.0)} & 67.8 & 68.4 & 67.7 \\
  Only \texttt{thinking}  & 50.1\,\mscriptsize{(3.4)} & 53.8\,\mscriptsize{(2.3)} & 50.6\,\mscriptsize{(2.7)} & 70.5 & 70.1 & 70.1 \\
              Only \texttt{territory}  & 68.2\,\mscriptsize{(1.8)} & 66.8\,\mscriptsize{(2.2)} & 66.6\,\mscriptsize{(2.3)} & 71.4 & 68.4 & 68.9 \\
\end{tabular}}
    %\caption{Scores (± std) for discrete emotion classification based on components.}
    \label{tab:pred_emotion}
\end{table}

\vspace{0.25cm}
\pause

%\centering
$\rightarrow$ Each component improves prediction performance, the best results are achieved by jointly considering all components

\vspace{0.25cm}
\pause

$\rightarrow$ Some components benefit from contextual understanding and world knowledge (\textit{e.g.}, \texttt{behavior} and \texttt{thinking})

\end{frame}

\begin{frame}{Quantitative analysis of dream narratives}

  %We apply our framework to dream narratives as they possess a narrative structure and represent attempts to communicate subjective experience

Need other datasets with narrative structure, emotional content, and available for research

\vspace{0.5cm}
\pause

Quantitative dream analysis most relies on dream databases (\textit{e.g.}, DreamBank composed of 27,000 dreams) and annotation schemes (\textit{e.g.}, Hall and Van de Castle system) \citep{domhoffStudyingDreamContent2008b}

\vspace{0.5cm}
\pause

The annotation process is complex and costly

\vspace{0.5cm}
\pause

How to automate the annotation process?

%side project in my PhD
    
\end{frame}

\begin{frame}{Character and emotion detection in dream narratives}

%\textbf{Goal}: Character and emotion detection in dream narratives

%\pause

\begin{figure}
    \centering
    \includegraphics[width=0.8\linewidth]{img/dream_method.png}
    %\caption{Codes describing characters and their emotions are converted into
%natural language to produce the training data.}
    \label{fig:placeholder}
\end{figure}

%\scriptsize

%\textbf{G. Cortal}. \href{https://aclanthology.org/2024.lrec-main.1282/}{Sequence-to-Sequence Language Models for Character and Emotion Detection in Dream Narratives}. \textit{LREC-COLING 2024}.
    
\end{frame}

\begin{frame}{Results}

\texttt{Baseline} is LaMini-Flan-T5 finetuned on 1823 dream narratives

\begin{table}
    \centering
    \resizebox{1.0\textwidth}{!}{
\begin{tabular}{lcccccc}
\textbf{Model} & \textbf{Status} & \textbf{Gender} & \textbf{Identity} & \textbf{Age} & \textbf{Character} & \textbf{Emotion} \\
\hline
\texttt{Baseline} & 82.87 & 78.02 & 76.17 & 86.21 & 64.74 &  75.13 \\
\hline
\texttt{No\textsubscript{semantics}} & 71.37 & 56.54\textbf{*} & 61.0  & 90.51  & 41.79\textbf{*} &  75.79 \\
\texttt{No\textsubscript{names}} & 80.66\textbf{*}  & 74.32\textbf{**}  & 74.2  & 83.95\textbf{*}  & 60.93\textbf{**} &  73.04\textbf{*} \\
\hline
\texttt{Size\textsubscript{small}} & 78.35\textbf{**}  & 72.13\textbf{**}  & 70.25\textbf{**}  & 81.66\textbf{**}  & 56.79\textbf{**} &  70.15\textbf{**} \\
\texttt{Size\textsubscript{large}} & 84.51\textbf{*} & 80.3\textbf{**}  & 78.63\textbf{**}  & 87.29  & 67.63\textbf{**} & 74.71 \\
\hline
\texttt{First\textsubscript{group}} & 82.33  & 77.71  & 74.86  & 85.61  & 63.71 &  71.94 \\
\texttt{First\textsubscript{individual}} & 80.59\textbf{**}  & 76.14  & 74.22\textbf{*}  & 83.87\textbf{**} & 62.67 &  67.32 \\
\texttt{First\textsubscript{emotion}} & 83.92 & 78.74 & 77.06  & 87.63  & 64.97 &  72.03 \\
\hline
\texttt{Conversion\textsubscript{comma}} & 84.02\textbf{**}  & 79.84\textbf{**}  & 77.67\textbf{**}  & 87.08\textbf{*}  & 66.69\textbf{**} &  73.68 \\
\texttt{Conversion\textsubscript{marker}} & 82.39  & 78.45  & 76.53  & 86.09  & 65.44 &  74.36 \\
\hline
\texttt{Cross-validation} & 86.28 & 81.9  & 79.51  & 89.52 & 68.64 &  76.18\\
\end{tabular}}
  \caption{$F_1$-scores for character and emotion detection. Significant differences from \texttt{baseline}: ** ($p<0.01$), * ($p<0.05$).}
  \label{tab:result}
\end{table}

%\vspace{0.25cm}
\pause

$\rightarrow$ Language models can effectively address character and emotion detection in dream narratives
    
\end{frame}

\begin{frame}{Oneirogen, a language model for dream generation}  

Oneirogen (\href{https://huggingface.co/gustavecortal/oneirogen-0.5B}{0.5}, \href{https://huggingface.co/gustavecortal/oneirogen-1.5B}{1.5}, \href{https://huggingface.co/gustavecortal/oneirogen-7B}{7B}), a language model for dream generation. It is based on \href{https://huggingface.co/Qwen/Qwen2-7B}{Qwen2} and was trained on \href{https://dreambank.net/}{DreamBank}

\vspace{0.25cm}
\pause

\noindent Oneirogen was used to produce \href{https://huggingface.co/datasets/gustavecortal/the-android-and-the-human}{The Android and The Machine}, an English dataset composed of 10,000 real and 10,000 generated dreams


\vspace{0.25cm}
%\pause

\textit{I'm in a building that seems to be a school or maybe a university. There is a lot of noise and activity, and everyone is very busy talking. It is very loud and unpleasant - too loud to talk to anyone easily. The walls are made out of some soft material that might be plastic foam.}

\vspace{0.25cm}

\textit{I was at a shop. There were lots of people there and I lost Mom and Ezra. Later, we were in a car park. We went to get pizza's for dinner from the nearby pizza place but it was really late so they wouldn't serve us. [I think I was also walking around the shops earlier].}

\end{frame}


\begin{frame}{}
\Large
\begin{center}
    Formalization of style in personal narratives
    \section{Formalization of style in personal narratives}
\end{center}

\vspace{1.5cm}

\footnotesize

\textbf{G. Cortal} and A. Finkel. \href{https://gustavecortal.com/data/Formalizing_Style_in_Personal_Narratives.pdf}{Formalizing Style in Personal Narratives}. \textit{EMNLP 2025}.
\end{frame}

\begin{frame}{How is subjective experience communicated in narratives?}

\pause

We use narratives to express our representations of reality and make sense of the world \citep{brunerActsMeaning1990}

\vspace{0.5cm}
\pause

In everyday usage, style refers to a distinctive manner of expression

\vspace{0.5cm}
\pause

We use style as a proxy to study how subjective experience is linguistically communicated

\vspace{0.5cm}
\pause

We narrow the general definition of style: \textit{a distinctive manner of communicating subjective experience in narratives}

\end{frame}

\begin{comment}

  \begin{frame}
{Related work on style analysis}

TODO

Stylometry, stylistics, knowledge graph, cognitive linguistics, etc.
\end{frame}

\end{comment}

\begin{frame}{How to give an operational definition of style?}

\pause

\textbf{Hypothesis}: An individual uses some redundant choices of features that characterize its style

\vspace{0.25cm}
\pause

%\textbf{Research task}: Formalize style as \textit{patterns of linguistic choices that encode subjective experience}

%\vspace{0.5cm}
%\pause

\begin{enumerate}[<+->]
    \item A sequence-based framework defining style as patterns in sequences of linguistic choices% grounded in systemic functional linguistics
    \item A methodology for identifying patterns using sequence analysis
    \item A case study on dream narratives %, showing how the analysis of patterns can reveal psychological insights
\end{enumerate}
    
\end{frame}

\begin{frame}{What linguistic features encode subjective experience?}

We ground our framework in \textit{systemic functional linguistics} \citep{hallidayIntroductionFunctionalGrammar2014a}

\pause
\vspace{0.5cm}

Meaning emerges through choices in systems of linguistic features to achieve communicative goals

\pause
\vspace{0.5cm}

Language achieves three functions: interpersonal (language builds social relationships), textual (information is organized to create coherent messages), and \textit{ideational} (language represents experience)

    
\end{frame}

\begin{frame}{What linguistic features encode subjective experience?}

\pause

Language represents experience through \textit{processes}, \textit{participants} and \textit{circumstances}

\pause

\begin{table}[!htb]
    \centering
    \resizebox{\textwidth}{!}{
    \begin{tabular}{p{4cm}|p{11cm}}
        %\hline
        \textbf{Processes} & \textbf{Examples} \\ \hline
        \texttt{Action}: actions and events in the physical world. &
        [He]$_{\text{Actor}}$ [\textbf{takes}]$_{\text{Action}}$ [the valuable]$_{\text{Affected}}$ \newline
        
        [Members of my cult]$_{\text{Actor}}$ [\textbf{have made]}$_{\text{Action}}$ [1500 euros]$_{\text{Result}}$ \newline
        
        [I]$_{\text{Actor}}$ [\textbf{give}]$_{\text{Action}}$ [her]$_{\text{Recipient}}$ [a chance]$_{\text{Range}}$ \\ \hline
        
        \texttt{Mental}: internal experiences such as thoughts, perceptions, and feelings. &
        [We]$_{\text{Senser}}$ [\textbf{believe}]$_{\text{Mental}}$ [women are the leaders of change]$_{\text{Phenomenon}}$ \newline
        
        [The moon]$_{\text{Senser}}$ [\textbf{sees}]$_{\text{Mental}}$ [the earth]$_{\text{Phenomenon}}$ \newline
        
        [He]$_{\text{Senser}}$ [\textbf{disliked}]$_{\text{Mental}}$ [Gilbert's writing]$_{\text{Phenomenon}}$ \\ \hline
        
        \texttt{Verbal}: acts of communication. &
        [David]$_{\text{Sayer}}$ [\textbf{said}]$_{\text{Verbal}}$ [``the corrupt, criminals and money launderers'']$_{\text{Verbiage}}$ \\ \hline

        \texttt{State}: states of being, having, or existence. &

         There [\textbf{was}]$_{\text{Existential}}$ [a swimming pool]$_{\text{Existent}}$ \newline
        
        [John]$_{\text{Carrier}}$ [\textbf{is}]$_{\text{State}}$ [an interesting teacher]$_{\text{Attribute}}$ \newline
        
        [Hadrian's Wall]$_{\text{Possessor}}$ [\textbf{has}]$_{\text{State}}$ [something for everyone]$_{\text{Possessed}}$ \\ %\hline
    \end{tabular}}
    %\caption{Processes with their participants.}
    \label{tab:process_participants}
\end{table}
    
\end{frame}

\begin{frame}{Pipeline for our sequence-based framework}


\begin{table}[!ht]
  \centering
  \small
  \renewcommand{\arraystretch}{1.1}
  \begin{threeparttable}
    %\caption{Illustrative pipeline for our sequence-based framework. We first segment “I wake in a dark room. I feel a cold wind. I tell myself to move.” into clauses, then identify features such as processes and participants for each clause. Each text is mapped to a symbolic sequence using an alphabet based on extracted features.}
    \label{tab:example}
    \begin{tabular}{lll}
      %\toprule
      \textbf{Clause} & \textbf{Process (symbol)} & \textbf{Participants} \\
      \midrule
      I wake in a dark room         & \texttt{Action} (\textbf{a})  & \texttt{Actor} \\
      I feel a cold wind            & \texttt{Mental} (\textbf{m})  & \texttt{Senser},\\
                                            &             & \texttt{Phenomenon} \\
      I tell myself to move         & \texttt{Verbal} (\textbf{v})  & \texttt{Sayer},\\
                                            &             & \texttt{Recipient} \\
      \bottomrule
    \end{tabular}

    \begin{tablenotes}[flushleft]
      \footnotesize
      \item \textbf{Sequence:} $amv$\quad|\quad
            \textbf{Substrings:} \{am, mv\}
    \end{tablenotes}
  \end{threeparttable}
\end{table}

\pause

\begin{enumerate}[<+->]
    \item We first segment \textit{“I wake in a dark room. I feel a cold wind. I tell myself to move.”} into clauses
    \item Identify features (\textit{e.g.}, processes and participants) for each clause using in-context learning with large language models
    \item Each narrative is mapped to a symbolic sequence using an alphabet based on identified features
    \item We perform sequence analysis to identify patterns such as frequent substrings and representative sequences
\end{enumerate}
    
\end{frame}

\begin{frame}{Conclusion}

How to model subjective experience in personal narratives?

\vspace{0.5cm}
\pause

\begin{itemize}[<+->]
    \item Definition of objectives and scope using cognitive science
    \item Construction of an emotion dataset 
    \item Training of language models for emotion analysis 
    \item Formalization of style in personal narratives
    %\item Automatic thematic analysis in mental health narratives% using language models
\end{itemize}

%\vspace{0.5cm}
%\pause

%\small

%\textit{My research models are publicly hosted on Hugging Face and were trained using the Jean Zay supercomputer}
    
\end{frame}

\begin{frame}{Perspectives}

  \begin{itemize}[<+->]
    \item Emotion analysis for mental health (empathic support, cognitive distortions, theory of mind)
    \item Psychology of language models (sycophancy, thought operations)
    \item Post-training for psychology (preferences and reasoning data)
  \end{itemize}

\vspace{0.5cm}
\pause

I co-supervised ongoing PhD thesis: Nicolas Richet (multimodal emotion recognition), Amine Haddou (cognitive distortions), and Raphael Faure (style analysis)
  
\end{frame}

\begin{frame}{Post-training for psychology}

Piaget (\href{https://huggingface.co/gustavecortal/Piaget-0.6B}{0.6}, \href{https://huggingface.co/gustavecortal/Piaget-1.7B}{1.7}, \href{https://huggingface.co/gustavecortal/Piaget-4B}{4}, \href{https://huggingface.co/gustavecortal/Piaget-8B}{8B}), a model for psychological reasoning

\smallskip
\pause

Domain filtering on open reasoning traces from \href{https://huggingface.co/datasets/cognitivecomputations/dolphin-r1}{Dolphin R1} and \href{https://huggingface.co/datasets/GeneralReasoning/GeneralThought-430K}{General Reasoning}

\smallskip
\pause

Prompts were embedded, clustered with $k$-means ($k=20\,000$) and majority-voted for domain labels using \href{https://huggingface.co/Qwen/Qwen3-1.7B}{Qwen3-1.7B}%, following the \href{https://huggingface.co/Intelligent-Internet/II-Medical-8B-1706}{Intelligent Internet pipeline}. 

\smallskip
\pause

Clusters tagged psychology or philosophy were retained for LoRA finetuning based on Qwen3% ($\text{rank}=8$, $\alpha=16$, $\text{max length}=2048$, $\text{epoch}=1$, $\text{batch size}=16$).

\smallskip

\begin{figure}
    \centering
    \includegraphics[scale=0.15]{img/lora.png}
    %\caption{Examples of reasoning traces used for post-training Piaget.}
    \label{fig:psychological_reasoning_examples}
\end{figure}

\end{frame}

\begin{frame}{Post-training for psychology}


Beck (\href{https://huggingface.co/gustavecortal/Beck-0.6B}{0.6}, \href{https://huggingface.co/gustavecortal/Beck-1.7B}{1.7}, \href{https://huggingface.co/gustavecortal/Beck-4B}{4}, \href{https://huggingface.co/gustavecortal/Beck-8B}{8B}), a model aligned with psychotherapeutic preferences

\smallskip
\pause

Beck is based on \href{https://huggingface.co/gustavecortal/Piaget-8B}{Piaget} and was finetuned on psychotherapeutic preferences from \href{https://huggingface.co/datasets/Psychotherapy-LLM/PsychoCounsel-Preference}{PsychoCounsel-Preference} using preference optimization (ORPO) and LoRA

\smallskip
\pause

\begin{figure}
    \centering
    \includegraphics[scale=0.17]{img/orpo.png}
    %\caption{Examples of reasoning traces used for post-training Piaget.}
    \label{fig:orpo}
\end{figure}


\end{frame}

\begin{frame}{}
\Large
\begin{center}
    Appendix
    \section{Appendix}
\end{center}

\end{frame}

\begin{frame}{Results on the war veteran}
  \begin{table}
    \centering
    \begin{tabular}{llccc}
%\toprule
\textbf{Group} & \textbf{Category} & \textbf{\% Vet} & \textbf{\% Total} & \textbf{$\Delta$} \\
\midrule
\multirow{5}{*}{Identity} & known* & 24.9 & 51.6 & -26.7 \\
& prominent & 1.9 & 2.5 & -0.6 \\
& occupational* & 22.4 & 8.0 & 14.4 \\
& ethnic* & 4.1 & 0.9 & 3.1 \\
& unknown* & 46.8 & 37.0 & 9.8 \\
\addlinespace
\multirow{4}{*}{Gender} & male* & 56.2 & 43.0 & 13.1 \\
& female* & 24.1 & 33.1 & -9.0 \\
& joint & 10.9 & 12.2 & -1.3 \\
& undefined & 7.9 & 8.7 & -0.9 \\
%\hline
\end{tabular}
\caption{Identity and gender proportions for the veteran (n=566 narratives) versus other dreamers. $\Delta$ shows the difference in percentage points. * indicates significant difference ($p < 0.05$).}
\label{tab:case_study_war_veteran}
\end{table}

\pause

$\rightarrow$ The war veteran dreams more about \textit{occupational}, \textit{ethnic}, and \textit{unknown} identities compared to other dreamers.

\end{frame}

\begin{frame}{Results on the war veteran}

We compare the proportion of sequences containing a given substring

\pause

\begin{figure}[!htb]
     \begin{subfigure}[b]{0.5\textwidth}
         \centering
         \includegraphics[scale=0.2]{img/viet_odds_2.png}
         \caption{Size 2.}
         \label{fig:viet_odds2}
     \end{subfigure}
         \begin{subfigure}[b]{0.4\textwidth}
         \centering
         \includegraphics[scale=0.2]{img/viet_odds_3.png}
         \caption{Size 3.}
         \label{fig:viet_odds3}
     \end{subfigure}
        \caption{Top substring odds ratio between the veteran and the norm}
        \label{fig:viet_odds}
\end{figure}

\pause


We show a preference for the war veteran to remain in a verbal process, as indicated by substrings such as \textit{verbal.verbal} and \textit{verbal.verbal.verbal} with high odds ratios (respectively 2.00 and 1.75)

\end{frame}

\begin{frame}{Results on the war veteran}

\begin{figure}
    \centering
    \includegraphics[width=0.65\linewidth]{img/dendogram_viet.png}
    \caption{Dendrogram with Ward linkage and cosine similarity}% on substrings of size one, two, and three.}
    \label{fig:dendogram}
\end{figure}

\vspace{0.25cm}
\pause

\textbf{Representative sequences}: \textit{savamasasaaamaaasavvvaaaaaaavssaaaaa} and \textit{sssssavaavssvsavvvvsmasasaasasaamaamvmsss} with $a=action, m=mental, s=state, v=verbal$

%\vspace{0.25cm}
%\pause

%Two templates: a highly action-oriented structure or a more varied structure alternating between state and action processes
    
\end{frame}

\begin{frame}{Selected research papers}
  \scriptsize

    \fullcitenourl{bonardImprovingLanguageModels2024a}

    \vspace{0.25cm}

    \fullcitenourl{cortalEmotionRecognitionBased2023d}

    \vspace{0.25cm}

    \fullcitenourl{cortalSequencetoSequenceLanguageModels2024}

    \vspace{0.25cm}

    \fullcitenourl{cortal-finkel-2025-formalizing}
\end{frame}

\begin{frame}{}
\Large
\begin{center}
    References
    \section{References}
\end{center}

\end{frame}

\begin{frame}[allowframebreaks]{References}
\printbibliography
\end{frame}

\end{document}
